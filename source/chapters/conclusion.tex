%%%%%%%%%%%%%%%%%%%%%%%%%%%%%%%%%%%%%%%%%%%%%%%%%%%%%%%%%%%%%%%%%%%%%%%%%%%%%%%
%%%%%%%%%%%%%%%%%%%%%%%%%%%%%%%%%%%%%%%%%%%%%%%%%%%%%%%%%%%%%%%%%%%%%%%%%%%%%%%
\chapter{Conclusion}
\label{chap:conclusion}
%%%%%%%%%%%%%%%%%%%%%%%%%%%%%%%%%%%%%%%%%%%%%%%%%%%%%%%%%%%%%%%%%%%%%%%%%%%%%%%
%%%%%%%%%%%%%%%%%%%%%%%%%%%%%%%%%%%%%%%%%%%%%%%%%%%%%%%%%%%%%%%%%%%%%%%%%%%%%%%

\section{Summary}
\label{sec:summary}

\begin{markdown}
-   a collection of open-source tools, interoperating
    -   python, lilypond, latex, abjad and consort,
-   a model of notation
    -   score as hierarchy (this is not contentious)
    -   a model of time: timespans, hierarchical rhythm, meter
    -   a model which is explicit, expressive (in the programming sense)
    -   which aspires to be compositionally agnostic
    -   (although it certainly must remain engaged with western notation)
    -   and which affords inspection, iteration, selection
-   a model of composition
    -   specification
    -   interpretation
    -   maquette
    -   segments
-   overarching concepts
    -   configuration, aggregation
\end{markdown}

The previous chapters have discussed a computational model of music
composition, implemented in the Python library Consort, and the model of
notation which it extends, Abjad. The various open-source systems -- \LaTeX{},
LilyPond, Python, etc. -- which interoperate to make these twin computational
models possible have also been demonstrated, and some standard solutions for
establishing a document preparation workflow which streamlines and accelerates
a cycle of score visualization has been proposed.

As described in \autoref{chap:a-model-of-notation}, Abjad's model of notation
treats musical score as a hierarchy consisting of containers -- staves, voices,
measures and tuplets -- and leaves -- notes, rests and chords --, to which
indicators -- clefs, dynamics, etc. -- and spanners -- slurs, beams, glissandi,
hairpins, and so forth -- can be attached. Abjad's model is clear and explicit
whenever possible. Those objects comprising a score which a composer might wish
to create -- what we might call the semantic content of the score\footnote{ As
opposed to those objects which are necessary or implicit, such as staff lines,
bar lines, measure numbers, etc. Of course, for some composers, staff lines can
and do represent semantic musical content. However, when creating input for
LilyPond, I would argue that staff lines are generally simply implicit. } --
are all represented by classes in Abjad, each with a well-defined interface
exposing only those properties and methods pertinent to that class. Abjad's
notation model strives for composition-process agnosticism\footnote{
Agnosticism here stretches only so far as being agnostic of all compositional
processes so long as they revolve around Western common practice notation. },
allowing composers to work directly with the elemental notation objects rather
than obligating them to rely on opinionated or idiosyncratic mechanisms. Abjad
provides a variety of models of musical time, discussed at length in
\autoref{chap:time-tools}, such as timespans and metrical hierarchies. These
time models permit alternative means of constructing, coordinating and
transforming musical structures than those provided by simply working with
score trees directly. Timespans, especially, afford the sketching of dense
polyphonic phrasing structures, and have been foundational to my working
process for years, explicitly since \emph{Aurora} and certainly with intention,
although not name, for many years prior to that. Although I and the other Abjad
developers have found timespans to be incredibly utilitarian, and certainly one
of the most fundamental tools in our toolkit for talking about time in score, I
initially\footnote{ Timespans as a compositional tool in Abjad began, in
spirit, with the \texttt{timeintervaltools} subpackage, my first large
contribution to Abjad, authored around 2010, which introduced a timespan-like
class called \texttt{TimeInterval} and a \texttt{TimeIntervalTree} for
containing them. These classes were named after the \enquote{interval-tree}
data structure, often used for modeling scheduling conflicts, as it provides a
highly-optimized search algorithm for overlap between time intervals and other
intervals or offsets. Trevor Ba\v{c}a later introduced a much more generalized
\texttt{timespantools} subpackage and nominative \texttt{Timespan} class, into
which I merged some of the more idiosyncratic functionality, such as timespan
explosion. } developed them as an affordance for structuring large-scale
orchestral works. All of these aspects, combined with Abjad's tools for
iterating over, selecting, and inspecting score components provides a strong
foundation for others to implement their own personal models of composition:
how one goes about organizing notation into a musical work.

For my part, Consort constitutes such a model of composition: a collection of
abstractions for organizing the elements of notation.

\section{Implications}
\label{sec:implications}

\begin{markdown}
-   implications for how one works:
    -   re-use
        -   abstraction and encapsulation define, in some sense, a voice
    -   extension
        -   simply because i haven't described something here doesn't mean i
            haven't thought of it
        -   in no small sense, all of this grows as a gradual extension of the
            notation model, itself growing from the previous projects, and from
            the underlying language
        -   there are always foot-holds for new work
        -   the "library-ness", "open-sourceness", and testing regime all act
            as bulwarks against disappearing, and as barrier-to-entry reducers
            for newcomers
    -   separation of notation and composition is not trivial
-   score as expression
    -   interpretation defines an algorithm
    -   that algorithm further extended by the insertion of additional
        sub-algorithms: handlers, rhythm-makers
-   patterns and randomness
    -   a sufficiently complex pattern of values is indistinguishable to a
        listener from a random sequence
    -   this is compounded when multiple sequences of different lengths
        interact
\end{markdown}

Coyly:

\begin{equation}
\displaystyle\sum_{i=1}^{n} Interpretation(Specification_i)
\end{equation}

\section{Lacunae \& future work}
\label{sec:lacunae-and-future-work}

\begin{markdown}
-   score-as-expression is delicate
-   multi-staff writing and pedaling
-   more flexible pitch structuring
-   idiomatic writing
-   explicit modeling of variation, transformation
-   simpler targeting of specific changes (this note, right here)
\end{markdown}

\section{Parting words}
\label{sec:parting-words}