%%%%%%%%%%%%%%%%%%%%%%%%%%%%%%%%%%%%%%%%%%%%%%%%%%%%%%%%%%%%%%%%%%%%%%%%%%%%%%%
%%%%%%%%%%%%%%%%%%%%%%%%%%%%%%%%%%%%%%%%%%%%%%%%%%%%%%%%%%%%%%%%%%%%%%%%%%%%%%%
\chapter{Introduction}
\label{chap:introduction}
%%%%%%%%%%%%%%%%%%%%%%%%%%%%%%%%%%%%%%%%%%%%%%%%%%%%%%%%%%%%%%%%%%%%%%%%%%%%%%%
%%%%%%%%%%%%%%%%%%%%%%%%%%%%%%%%%%%%%%%%%%%%%%%%%%%%%%%%%%%%%%%%%%%%%%%%%%%%%%%

\begin{markdown}

# What is this?

Laying out the terms.

This is an analysis of a model of composition, implemented computationally.

This is not an analysis of any works, although works are included.

This is not a survey of techniques used by composers working in
computer-assisted composition, but the work of one composer implementing for
the specifics of his own process. This work may well be applicable to others in
many ways, but I do not claim universality.

Provide a description of formalized score control.

Mention Abjad, Python, LilyPond and LaTeX very briefly. Drop names as
appropriate.

My aim is to provide a sufficiently detailed explanation of the theory behind
this work as well as my specific implementation that those who read
this would be able to consult the source to each of the three Invisible Cities
scores included in the appendices and make some (or a lot of) sense of them.

It is then, more of a tutorial than anything else.

# Background

Both historical and personal.

Provide wider discussion of Abjad, Python, LilyPond and LaTeX.

Differentiate from Max, PWGL, OM.

# Overview of the dissertation

The first six chapters...

Chapter 2 presents an overview of Abjad, detailing its structure and usage in
the creation of scores.

Chapter 3 expands on chapter 2, discussing various models of musical time in
detail, and introducing many of tools and techniques employed in Consort to
create large-scale musical works.

Chapter 4 analyzes the mechanisms implemented in Consort to specify the
structure of scores at a high level and to interpret those specifications in
order to produce notation.

Chapter 5 discusses practical concerns surrounding the composition of scores in
software including project layout, typesetting workflows, version control and
testing.

Chapter 6, the conclusion to the prose portion of the dissertation, summarizes
the previously presented research, and suggests implications and future work.

The remaining chapters consist of five scores, all composed computationally
with Python and LilyPond.

Chapter 7, Aurora/Mbrsi

- my first formal research into composition with timespans

Chapter 8, Plague Water

- my first attempt at composing scores in segments, implemented with the
precursor to Consort

Chapters 9, 10 and 11 present a set of three pieces, Invisible Cities (i):
Zaira, Invisible Cities (ii): Armilla and Invisible Cities (iii): Ersilia, all
implemented via Consort.

The appendices contain the source to all classes and functions implemented in
Consort, as well as the source to all material and segment definitions, as well
as any LilyPond stylesheets, for the three Invisible Cities scores.

\end{markdown}