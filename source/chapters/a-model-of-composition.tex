%%%%%%%%%%%%%%%%%%%%%%%%%%%%%%%%%%%%%%%%%%%%%%%%%%%%%%%%%%%%%%%%%%%%%%%%%%%%%%%
%%%%%%%%%%%%%%%%%%%%%%%%%%%%%%%%%%%%%%%%%%%%%%%%%%%%%%%%%%%%%%%%%%%%%%%%%%%%%%%
\chapter{\emph{Consort}: a model of composition}
\label{chap:a-model-of-composition}
%%%%%%%%%%%%%%%%%%%%%%%%%%%%%%%%%%%%%%%%%%%%%%%%%%%%%%%%%%%%%%%%%%%%%%%%%%%%%%%
%%%%%%%%%%%%%%%%%%%%%%%%%%%%%%%%%%%%%%%%%%%%%%%%%%%%%%%%%%%%%%%%%%%%%%%%%%%%%%%

\todo[inline]{\textbf{TODO:} Explain motivation for Consort.}

\todo[inline]{\textbf{TODO:} Explain music specifier seeds.}

\begin{comment}
<abjad>[hide=true]
import collections
import consort
</abjad>
\end{comment}

\begin{comment}
\begin{markdown}
-   Materials
-   Configuration
-   Templating
-   Layers
-   Coarse to fine
-   Rhythm first
-   What is specification? What is a specifier? What is configuration and
    aggregation?
-   What should happen musically, where should it happen?
-   What is material?
-   What is music?
-   Rhythm is interpreted first, as all other parameters depend on it.
-   Rhythm is interpreted from "coarse" to "fine": from the level of phrase
    boundaries to the level of individual notes, rests, tuplets and ties.
-   This discussion only focuses on notation, nothing related to aesthetic
    experience, physical modeling or anything else. This is a tool for a
    specific composer to create scores, not a discussion explicitly of why they
    would work this way (although that should be discussed in the conclusion).
-   Specification and interpretation conceive of the score as a single, hugely
    complex expression.
-   Templating as variation.
-   Define what composition means here: laying out symbols on the page.
-   This way of thinking and working does not attempt to define or even model
    concepts like "melody" or even "phrase". They're too vague. If we use that
    terminology at all, it is only in the most incredibly constrained way.
\end{markdown}
\end{comment}

Consort, a Python library I have written as an extension to Abjad, models the
process of composing of notated musical scores in terms of a repeated cycle
containing three distinct stages: \emph{specification}, \emph{interpretation}
and \emph{visualization}. What follows is a detailed analysis of the various
algorithms and subroutines employed during Consort's specification and
interpretation stages. Visualization, via Abjad's \emph{illustration protocol}
was already discussed in \autoref{ch:a-model-of-notation}.

\section{Specification}
\label{sec:specification}

\emph{Specification} describes how \emph{out-of-time materials} -- both
concrete and programmatic -- should be deployed \emph{in-time} in a
\emph{segment} of musical score as notation. Materials encompass abstractions
-- such as pitch sets or collections of performance technique indications --,
concrete fragments of \emph{music} -- narrowly defined here as any contiguous
selection of score components --, and procedures for producing, altering or
embellishing music such as rhythm-makers or attachment-handlers. Score segments
comprise any contiguous passage of music, demarcating an area of compositional
concern. Consort treats scores as comprised of at least one segment, but
potentially many more concatenated together. Any segment may of course contain
arbitrarily complex inner structuring. Separation of scores into distinct
segments acts then mainly as an aid for the composer, both by simplifying the
complexity of the current specification under consideration, and by allowing
the typesetting engine -- LilyPond -- to display more manageable amounts of
notation than the full score, thus speeding up the cycle of specifying,
interpreting and visualizing.

\subsection{Segment-makers}
\label{ssec:segment-makers}

Composers specify segments by creating and progressively configuring
\emph{segment-makers}, classes which conceptually mirror the rhythm- and
timespan-makers described in \autoref{chap:time-tools}, but on a much larger
scale. Such configuration parameters include tempo, permitted meters, desired
duration, and score template. Score templates, as outlined in
\autoref{ssec:score-templates}, are notation factories which build scores
comprised of staff groups, staves, voices, clefs and instruments, as necessary
to model the context hierarchy of a score to which no count-time components
have yet been added. All segment-makers in a score project must use the same
score template if they are to appear contiguously in the complete typeset
score. LilyPond will automatically concatenate scores with identical context
hierarchies. All contexts need to be present in every concatenated segment,
otherwise LilyPond will concatenate incorrectly. However, various typographic
overrides can be employed to make it seem that a context has disappeared.

The following defines a segment-maker with desired duration of 9 seconds, only
3/4 meters permitted, a score-template comprised of two rhythmic staves, and a
tempo or quarter-equals-60:

\begin{comment}
<abjad>
segment_maker = consort.SegmentMaker(
    desired_duration_in_seconds=9,
    permitted_time_signatures=[(3, 4)],
    score_template=templatetools.GroupedRhythmicStavesScoreTemplate(
        staff_count=2,
        with_clefs=True,
        ),
    tempo=indicatortools.Tempo((1, 4), 60),
    )
</abjad>
\end{comment}

\begin{abjadbookoutput}
\begin{singlespacing}
\vspace{-0.5\baselineskip}
\begin{lstlisting}
>>> segment_maker = consort.SegmentMaker(
...     desired_duration_in_seconds=9,
...     permitted_time_signatures=[(3, 4)],
...     score_template=templatetools.GroupedRhythmicStavesScoreTemplate(
...         staff_count=2,
...         with_clefs=True,
...         ),
...     tempo=indicatortools.Tempo((1, 4), 60),
...     )
\end{lstlisting}
\end{singlespacing}
\end{abjadbookoutput}

\noindent This segment-maker can be illustrated via \texttt{show()}, like many
other objects in Abjad. Illustration here invokes the segment-maker's
interpretation stage. The \texttt{verbose=False} flag prevents it from
printing a considerable amount of diagnostic information:

\begin{comment}
<abjad>[stylesheet=../consort.ily]
show(segment_maker, verbose=False)
</abjad>
\end{comment}

\begin{abjadbookoutput}
\begin{singlespacing}
\vspace{-0.5\baselineskip}
\begin{lstlisting}
>>> show(segment_maker, verbose=False)
\end{lstlisting}
\noindent\includegraphics[max width=\textwidth,]{assets/lilypond-5551103a6156a0c950aaf871d1206d96.pdf}
\end{singlespacing}
\end{abjadbookoutput}

\noindent By changing the tempo from quarter-equals-60 to quarter-equals-20,
the overall notated duration of the segment shrinks by two thirds, but the
duration in seconds remains the same. This mechanism allows segments to be
planned relative one another in terms of their \enquote{actual} durations:

\begin{comment}
<abjad>[stylesheet=../consort.ily]
slower_segment_maker = new(
    segment_maker,
    tempo=indicatortools.Tempo((1, 4), 20),
    )
show(slower_segment_maker, verbose=False)
</abjad>
\end{comment}

\begin{abjadbookoutput}
\begin{singlespacing}
\vspace{-0.5\baselineskip}
\begin{lstlisting}
>>> slower_segment_maker = new(
...     segment_maker,
...     tempo=indicatortools.Tempo((1, 4), 20),
...     )
>>> show(slower_segment_maker, verbose=False)
\end{lstlisting}
\noindent\includegraphics[max width=\textwidth,]{assets/lilypond-dac8ffa35eb4dae93ba418fbf1eaf390.pdf}
\end{singlespacing}
\end{abjadbookoutput}

\noindent Most importantly, segment-makers may be configured with any number of
\emph{music settings}, which object-model both \emph{when} and in \emph{which}
voices musical materials should be deployed.

\subsection{Music settings}
\label{ssec:music-settings}

Music settings represent a layer of musical texture, in one or more voices, of
arbitrary length. Music settings aggregate a timespan-maker, a target timespan
and any number of \emph{music specifiers}. The timespan-maker provides the
overall phrasing and density structure, the optional timespan identifier
defines in what portion of the current segment the timespan-maker's texture
should be spooled out, and the music specifiers define both in which voices the
timespan texture should appear as well as how those timespans should ultimately
be rendered as notation. The order in which composers configure segment-makers
with music settings defines each music setting's \emph{layer} -- the first
setting defined being layer 0, the second layer 1 and so forth, with each
higher layer number indicating higher precedence or \enquote{foregroundness}
--, determining how overlapping events in a single voice will mask one another.
The timespans created by music settings which are defined later during segment
specification \enquote{hide} any timespans created by those music settings
defined earlier. Score materials, including music settings, music specifiers,
timespan-makers and any other class pertinent to score creation -- potentially
even other segment-makers, may be defined from scratch in the same code module
as the segment-maker currently being configured, templated from another
material, or simply imported into the segment definition's namespace.

\subsection{Music specifiers}
\label{ssec:music-specifiers}

Music specifiers bundle all of the information necessary for a segment-maker to
generate the notational content for a sequence of one or more divisions,
grouped as a phrase. This information includes optional rhythm-maker
(\autoref{sec:rhythm-makers}), \emph{grace-handler}
(\autoref{ssec:grace-handlers}), \emph{pitch-handler}
(\autoref{ssec:pitch-handlers}) and \emph{attachment-handler}
(\autoref{ssec:attachment-handlers}) definitions -- all classes which describe
strategies for creating or modifying notation --, as well as a variety of other
properties including an optional \emph{minimum phrase duration} -- described
further in \autoref{ssec:splitting-pruning-and-consolidation} --, and
\emph{seed}.

Music specifiers can also be configured with a \emph{labels}: a tuple of one or
more arbitrary strings, identifying some quality of those performed timespans.
As demonstrated in \autoref{ssec:dependent-timespan-makers}, composers
configure dependent timespan-makers in order to create timespans according to
the disposition of performed timespans associated with specific voices.
Dependent timespan-makers can also be configured to select depended-upon
timespans based on the \texttt{labels} property of those performed timespans'
music specifiers, allowing an additional category for filtering. This helps
model creating a pedal voice mirroring not simply a pianist's
left- and right-hand events, but only those that actually depress keys,
ignoring guero events or other off-the-key percussive techniques for which no
pedaling is desired.

Consider the following timespan inventory, populated with performed timespans
associated with one of two voices, and annotated with music specifiers which
are either labeled or not:

\begin{comment}
<abjad>
unlabeled_music_specifier = consort.MusicSpecifier()
labeled_music_specifier = consort.MusicSpecifier(labels=['labeled'])
timespan_inventory = timespantools.TimespanInventory([
    consort.PerformedTimespan(
        layer=1,
        start_offset=0,
        stop_offset=4,
        music_specifier=labeled_music_specifier,
        voice_name='Voice 1',
        ),
    consort.PerformedTimespan(
        layer=1,
        start_offset=2,
        stop_offset=7,
        music_specifier=labeled_music_specifier,
        voice_name='Voice 2',
        ),
    consort.PerformedTimespan(
        layer=2,
        start_offset=6,
        stop_offset=8,
        music_specifier=unlabeled_music_specifier,
        voice_name='Voice 1',
        ),
    consort.PerformedTimespan(
        layer=2,
        start_offset=10,
        stop_offset=(25, 2),
        music_specifier=unlabeled_music_specifier,
        voice_name='Voice 2',
        ),
    consort.PerformedTimespan(
        layer=1,
        start_offset=11,
        stop_offset=14,
        music_specifier=labeled_music_specifier,
        voice_name='Voice 1',
        ),
    consort.PerformedTimespan(
        layer=1,
        start_offset=14,
        stop_offset=16,
        music_specifier=labeled_music_specifier,
        voice_name='Voice 1',
        ),
    consort.PerformedTimespan(
        layer=1,
        start_offset=15,
        stop_offset=16,
        music_specifier=labeled_music_specifier,
        voice_name='Voice 2',
        ),
    ])
show(timespan_inventory, key='voice_name')
</abjad>
\end{comment}

\begin{abjadbookoutput}
\begin{singlespacing}
\vspace{-0.5\baselineskip}
\begin{lstlisting}
>>> unlabeled_music_specifier = consort.MusicSpecifier()
>>> labeled_music_specifier = consort.MusicSpecifier(labels=['labeled'])
>>> timespan_inventory = timespantools.TimespanInventory([
...     consort.PerformedTimespan(
...         layer=1,
...         start_offset=0,
...         stop_offset=4,
...         music_specifier=labeled_music_specifier,
...         voice_name='Voice 1',
...         ),
...     consort.PerformedTimespan(
...         layer=1,
...         start_offset=2,
...         stop_offset=7,
...         music_specifier=labeled_music_specifier,
...         voice_name='Voice 2',
...         ),
...     consort.PerformedTimespan(
...         layer=2,
...         start_offset=6,
...         stop_offset=8,
...         music_specifier=unlabeled_music_specifier,
...         voice_name='Voice 1',
...         ),
...     consort.PerformedTimespan(
...         layer=2,
...         start_offset=10,
...         stop_offset=(25, 2),
...         music_specifier=unlabeled_music_specifier,
...         voice_name='Voice 2',
...         ),
...     consort.PerformedTimespan(
...         layer=1,
...         start_offset=11,
...         stop_offset=14,
...         music_specifier=labeled_music_specifier,
...         voice_name='Voice 1',
...         ),
...     consort.PerformedTimespan(
...         layer=1,
...         start_offset=14,
...         stop_offset=16,
...         music_specifier=labeled_music_specifier,
...         voice_name='Voice 1',
...         ),
...     consort.PerformedTimespan(
...         layer=1,
...         start_offset=15,
...         stop_offset=16,
...         music_specifier=labeled_music_specifier,
...         voice_name='Voice 2',
...         ),
...     ])
>>> show(timespan_inventory, key='voice_name')
\end{lstlisting}
\noindent\includegraphics[max width=\textwidth,]{assets/lilypond-8ff4b7fe83d5db8fb0d01627be42da4b.pdf}
\end{singlespacing}
\end{abjadbookoutput}

\noindent A dependent timespan-maker, configured to select timespans associated
with \enquote{Voice 1} and \enquote{Voice 2} produces dependent timespans in
the manner illustrated in \autoref{ssec:dependent-timespan-makers}:

\begin{comment}
<abjad>
dependent_timespan_maker = consort.DependentTimespanMaker(
    include_inner_starts=True,
    voice_names=('Voice 1', 'Voice 2'),
    )
result = dependent_timespan_maker(
    layer=3,
    music_specifiers={'Voice 3': None},
    timespan_inventory=timespan_inventory[:],
    )
show(result, key='voice_name')
</abjad>
\end{comment}

\begin{abjadbookoutput}
\begin{singlespacing}
\vspace{-0.5\baselineskip}
\begin{lstlisting}
>>> dependent_timespan_maker = consort.DependentTimespanMaker(
...     include_inner_starts=True,
...     voice_names=('Voice 1', 'Voice 2'),
...     )
>>> result = dependent_timespan_maker(
...     layer=3,
...     music_specifiers={'Voice 3': None},
...     timespan_inventory=timespan_inventory[:],
...     )
>>> show(result, key='voice_name')
\end{lstlisting}
\noindent\includegraphics[max width=\textwidth,]{assets/lilypond-cacf84e69a9aeb754e2908597bc776a2.pdf}
\end{singlespacing}
\end{abjadbookoutput}

\noindent Reconfiguring the above dependent timespan-maker to additionally
filter timespans whose music specifier is labeled \enquote{labeled} produces a
more restricted output. Note that the unlabeled 6/1-8/1 timespan in
\enquote{Voice 1} and the 10/1-25/2 timespan in \enquote{Voice 2} are ignored:

\begin{comment}
<abjad>
dependent_timespan_maker = new(
    dependent_timespan_maker,
    labels=['labeled'],
    )
result = dependent_timespan_maker(
    layer=3,
    music_specifiers={'Voice 3': None},
    timespan_inventory=timespan_inventory[:],
    )
show(result, key='voice_name')
</abjad>
\end{comment}

\begin{abjadbookoutput}
\begin{singlespacing}
\vspace{-0.5\baselineskip}
\begin{lstlisting}
>>> dependent_timespan_maker = new(
...     dependent_timespan_maker,
...     labels=['labeled'],
...     )
>>> result = dependent_timespan_maker(
...     layer=3,
...     music_specifiers={'Voice 3': None},
...     timespan_inventory=timespan_inventory[:],
...     )
>>> show(result, key='voice_name')
\end{lstlisting}
\noindent\includegraphics[max width=\textwidth,]{assets/lilypond-9382d23b443fa753fb33e78d0e7a805f.pdf}
\end{singlespacing}
\end{abjadbookoutput}

\noindent Music specifiers can be grouped into sequences called,
unsurprisingly, music specifier sequences, allowing a music setting to specify
that the timespans it creates associated with a certain voice should be
annotated with multiple different music specifiers in a patterned way:

\begin{comment}
<abjad>
music_specifier_sequence = consort.MusicSpecifierSequence(
    music_specifiers=['A', 'B', 'C'],
    )
</abjad>
\end{comment}

\begin{abjadbookoutput}
\begin{singlespacing}
\vspace{-0.5\baselineskip}
\begin{lstlisting}
>>> music_specifier_sequence = consort.MusicSpecifierSequence(
...     music_specifiers=['A', 'B', 'C'],
...     )
\end{lstlisting}
\end{singlespacing}
\end{abjadbookoutput}

\noindent Recall from \autoref{ssec:talea-timespan-makers} that talea
timespan-makers can create contiguous groups of 1 or more timespan associated
with a specific voice. Music specifier sequences can be used to annotated a
different music specifier to each timespan in a contiguous group, or to
annotated a different music specifier to \emph{each} contiguous group. This
behavior is controlled via the music specifier's \emph{application rate}
property:

\begin{comment}
<abjad>
music_specifiers = {'Voice': music_specifier_sequence}
target_timespan = timespantools.Timespan(0, (7, 4))
timespan_maker = consort.TaleaTimespanMaker(
    playing_groupings=(3,),
    )
timespan_inventory = timespan_maker(
    music_specifiers=music_specifiers,
    target_timespan=target_timespan,
    )
print(format(timespan_inventory))
</abjad>
\end{comment}

\begin{abjadbookoutput}
\begin{singlespacing}
\vspace{-0.5\baselineskip}
\begin{lstlisting}
>>> music_specifiers = {'Voice': music_specifier_sequence}
>>> target_timespan = timespantools.Timespan(0, (7, 4))
>>> timespan_maker = consort.TaleaTimespanMaker(
...     playing_groupings=(3,),
...     )
>>> timespan_inventory = timespan_maker(
...     music_specifiers=music_specifiers,
...     target_timespan=target_timespan,
...     )
>>> print(format(timespan_inventory))
timespantools.TimespanInventory(
    [
        consort.tools.PerformedTimespan(
            start_offset=durationtools.Offset(0, 1),
            stop_offset=durationtools.Offset(1, 4),
            music_specifier='A',
            voice_name='Voice',
            ),
        consort.tools.PerformedTimespan(
            start_offset=durationtools.Offset(1, 4),
            stop_offset=durationtools.Offset(1, 2),
            music_specifier='A',
            voice_name='Voice',
            ),
        consort.tools.PerformedTimespan(
            start_offset=durationtools.Offset(1, 2),
            stop_offset=durationtools.Offset(3, 4),
            music_specifier='A',
            voice_name='Voice',
            ),
        consort.tools.PerformedTimespan(
            start_offset=durationtools.Offset(1, 1),
            stop_offset=durationtools.Offset(5, 4),
            music_specifier='B',
            voice_name='Voice',
            ),
        consort.tools.PerformedTimespan(
            start_offset=durationtools.Offset(5, 4),
            stop_offset=durationtools.Offset(3, 2),
            music_specifier='B',
            voice_name='Voice',
            ),
        consort.tools.PerformedTimespan(
            start_offset=durationtools.Offset(3, 2),
            stop_offset=durationtools.Offset(7, 4),
            music_specifier='B',
            voice_name='Voice',
            ),
        ]
    )
\end{lstlisting}
\end{singlespacing}
\end{abjadbookoutput}

\noindent Changing the application rate from the default value of
\enquote{phrase} to \enquote{division} causes a different music specifier from
the sequence to be annotated to each timespan, rather than each contiguous
group of timespans:

\begin{comment}
<abjad>
music_specifier_sequence = new(
    music_specifier_sequence,
    application_rate='division',
    )
music_specifiers = {'Voice': music_specifier_sequence}
timespan_inventory = timespan_maker(
    music_specifiers=music_specifiers,
    target_timespan=target_timespan,
    )
print(format(timespan_inventory))
</abjad>
\end{comment}

\begin{abjadbookoutput}
\begin{singlespacing}
\vspace{-0.5\baselineskip}
\begin{lstlisting}
>>> music_specifier_sequence = new(
...     music_specifier_sequence,
...     application_rate='division',
...     )
>>> music_specifiers = {'Voice': music_specifier_sequence}
>>> timespan_inventory = timespan_maker(
...     music_specifiers=music_specifiers,
...     target_timespan=target_timespan,
...     )
>>> print(format(timespan_inventory))
timespantools.TimespanInventory(
    [
        consort.tools.PerformedTimespan(
            start_offset=durationtools.Offset(0, 1),
            stop_offset=durationtools.Offset(1, 4),
            music_specifier='A',
            voice_name='Voice',
            ),
        consort.tools.PerformedTimespan(
            start_offset=durationtools.Offset(1, 4),
            stop_offset=durationtools.Offset(1, 2),
            music_specifier='B',
            voice_name='Voice',
            ),
        consort.tools.PerformedTimespan(
            start_offset=durationtools.Offset(1, 2),
            stop_offset=durationtools.Offset(3, 4),
            music_specifier='C',
            voice_name='Voice',
            ),
        consort.tools.PerformedTimespan(
            start_offset=durationtools.Offset(1, 1),
            stop_offset=durationtools.Offset(5, 4),
            music_specifier='B',
            voice_name='Voice',
            ),
        consort.tools.PerformedTimespan(
            start_offset=durationtools.Offset(5, 4),
            stop_offset=durationtools.Offset(3, 2),
            music_specifier='C',
            voice_name='Voice',
            ),
        consort.tools.PerformedTimespan(
            start_offset=durationtools.Offset(3, 2),
            stop_offset=durationtools.Offset(7, 4),
            music_specifier='A',
            voice_name='Voice',
            ),
        ]
    )
\end{lstlisting}
\end{singlespacing}
\end{abjadbookoutput}

\noindent Consort provides an additional variation of music specifier called a
\emph{composite music specifier}.
Composite music specifiers allow for the definition of music for voices in
tandem, such as the fingering- and bowing-voice in a split-hands string
notation. When passed as part of a voice-name-to-music-specifier mapping to a
timespan-maker, that timespan-maker will create timespans for the first voice
of the composite music specifier and then create timespans for the second voice
as though it had its own dependent timespan-maker solely for those timespans.
This behavior ensures a degree of synchronization between pairs of voices which
should always appear at the same time in a score:

\begin{comment}
<abjad>
composite_music_specifier = consort.CompositeMusicSpecifier(
    primary_music_specifier='one',
    primary_voice_name='Viola 1 RH',
    rotation_indices=(0, 1, -1),
    secondary_voice_name='Viola 1 LH',
    secondary_music_specifier=consort.MusicSpecifierSequence(
        application_rate='phrase',
        music_specifiers=['two', 'three', 'four'],
        ),
    )
music_specifiers = {'Viola 1': composite_music_specifier}
timespan_inventory = timespan_maker(
    music_specifiers=music_specifiers,
    target_timespan=target_timespan,
    )
show(timespan_inventory, key='voice_name')
</abjad>
\end{comment}

\begin{abjadbookoutput}
\begin{singlespacing}
\vspace{-0.5\baselineskip}
\begin{lstlisting}
>>> composite_music_specifier = consort.CompositeMusicSpecifier(
...     primary_music_specifier='one',
...     primary_voice_name='Viola 1 RH',
...     rotation_indices=(0, 1, -1),
...     secondary_voice_name='Viola 1 LH',
...     secondary_music_specifier=consort.MusicSpecifierSequence(
...         application_rate='phrase',
...         music_specifiers=['two', 'three', 'four'],
...         ),
...     )
>>> music_specifiers = {'Viola 1': composite_music_specifier}
>>> timespan_inventory = timespan_maker(
...     music_specifiers=music_specifiers,
...     target_timespan=target_timespan,
...     )
>>> show(timespan_inventory, key='voice_name')
\end{lstlisting}
\noindent\includegraphics[max width=\textwidth,]{assets/lilypond-8366833a47e42f51e9db864d5c6ad709.pdf}
\end{singlespacing}
\end{abjadbookoutput}

\section{Rhythmic interpretation}
\label{sec:rhythmic-interpretation}

At any point during specification, a segment-maker may be interpreted to
produce an illustration. Score interpretation proceeds conceptually much like
compilation in classical computing, where a compiler parses an instruction set
written in some source language into an intermediate representation and then
transforms that same intermediate representation into instructions executable
on a target platform. In Consort's interpretation stage, the compiler is the
segment-maker itself, and the source instruction set its configuration -- its
tempo, permitted meters, music settings and so forth. Timespan inventories
produced by each music setting's timespan-maker, populated with timespans
annotated with music specifiers perform the role of the intermediate
representation. This intermediate representation acts as a \emph{maquette},
blocking out where in the resulting score segment various materials should be
deployed. The target of score interpretation is, unsurprisingly, a fully-fledge
score aggregated from Abjad score components. Interpretation takes place in two
broad stages -- rhythmic interpretation, followed by non-rhythmic
interpretation -- with the first stage producing a score populated solely with
rhythmic information, and the second stage applying grace notes, pitches,
indicators, spanners and various typographic overrides to the
previously-constructed rhythmic skeleton.

Seen from a high level, rhythmic interpretation proceeds from coarse- to
fine-grained. The segment-maker creates a *maquette* -- a model of the locations
of musical materials in the score -- by calling each of its
music-settings in turn to populate a timespan inventory. It then resolves
overlap conflicts within that inventory, fits meters against the resolved
inventory's offsets, splits and prunes the contents of the inventory according
to its fitted metrical structure, and finally converts the finished timespan
maquette into an actual score. This process, like interpretation overall, can
be roughly divided into work flows of \emph{maquette creation} and \emph{music
creation}, although in practice the two flows are interleaved significantly as
they actually influence one another. When creating the maquette, music settings
with \emph{independent} timespan-makers -- those which do not depend on the
contents of a previously created timespan inventory, specifically flooded and
talea timespan-makers -- are called in a first pass, and those with {dependent}
timespan-makers in a second. These two passes only differ significantly in that
meters are fitted against the segment's timespan maquette during the
independent timespan-maker pass, but not during the dependent.

\subsection{Populating the maquette}
\label{ssec:populating-the-maquette}

To populate the maquette, the segment-maker calls each of its music settings to
produce timespans according to their configured timespan-makers,
\emph{timespan-identifiers} -- optional specifications of which portion of the
segment's overall timespan to operate within -- and voice-associated music
specifiers. Timespan identifiers may include timespans, inventories of
timespans, or even expressions callable against the segment-makers own timespan
which evaluate to an inventory of timespans.

Music settings exist without any reference to a segment-maker, its desired
duration -- and therefore desired timespan --, or its score template. In order
to know which target timespan or timespans a music setting's timespan-maker
should operate within -- in the case of procedural timespan identifiers such as
\emph{ratio-parts expressions} which must be called against a preexisting
timespan in order to determine what part or parts of that timespan to use --
the music setting must resolve its timespan identifier against the segment's
desired duration. Target timespan resolution must also take into account offset
quantization, as the target timespans resulting from the evaluation of a
ratio-parts expression may not align against a power-of-two-denominator offset
grid such as 1/8, 1/16 or 1/32. Because timespan-makers produce their output
relative to the start-offset of their target timespan, a misaligned target
timespan -- starting at an offset like 1/3 or 15/7 rather than 1/4 or 0 -- will
cause all generated timespans to be misaligned.

Music setting's associate their music specifiers with strings containing
voice-name abbreviations. These abbreviations are always underscore-delimited
strings such as \texttt{violin\_1} or \texttt{piano\_lh} -- necessitated by
Python's keyword argument syntax so that they can be used as keys during class
instantiation -- which represent voices in a score, without having established
a concrete reference to those voice contexts. In order to match its music
specifiers against actual voice contexts in a score, the music setting must
resolve its voice-name abbreviations against a score template, looking up each
abbreviation on the template and returning the real name of the associated
context. This lookup process allows music settings to construct well-formed
voice-name-to-music-specifier mappings, implemented as ordered dictionaries and
ordered by the actual *score index* -- effectively, the vertical location -- of
each looked-up context in the segment-maker's under-construction score. As
demonstrated in \autoref{sec:timespan-makers}, timespan-makers require these
mappings to produce their output. Additionally, voice-name resolution
guarantees that the values in the resolved voice-name-to-music-specifier
mapping are always either a \texttt{MusicSpecifierSequence} or
\texttt{CompositeMusicSpecifier} instance via coercion, where any composite
music-specifier's primary and secondary music specifiers are themselves coerced
into music specifier sequences. This coercion ensures that all arguments to the
music setting's timespan-maker are in a well-formed and predictable state.
\footnote{Timespan-makers actually delegate the creation of performed and
silent timespans to music specifier sequences. While not demonstrated
explicitly in \autoref{sec:timespan-makers}, this delegation allows
timespan-makers to use both music specifier sequences and composite music
specifiers interchangeably, with the former creating timespans associated with
one voice and the later with two. When the values of a timespan-maker's input
voice-name-to-music-specifier mapping are neither music specifier sequences nor
composite music specifiers -- as was the case in all of the examples in
\autoref{sec:timespan-makers} -- they implicitly coerce those values into music
specifier sequences.}

\begin{comment}
<abjad>
music_setting = consort.MusicSetting(
    timespan_identifier=consort.RatioPartsExpression(
        parts=(0, 2),
        ratio=(1, 3, 2),
        ),
    timespan_maker=consort.FloodedTimespanMaker(),
    violin_2_lh='A',
    viola_lh=('B', 'C', 'D'),
    cello=consort.CompositeMusicSpecifier(
        primary_music_specifier='one',
        secondary_music_specifier=consort.MusicSpecifierSequence(
            music_specifiers=['two', 'three', 'four'],
            ),
        )
    )
</abjad>
\end{comment}

\begin{abjadbookoutput}
\begin{singlespacing}
\vspace{-0.5\baselineskip}
\begin{lstlisting}
>>> music_setting = consort.MusicSetting(
...     timespan_identifier=consort.RatioPartsExpression(
...         parts=(0, 2),
...         ratio=(1, 3, 2),
...         ),
...     timespan_maker=consort.FloodedTimespanMaker(),
...     violin_2_lh='A',
...     viola_lh=('B', 'C', 'D'),
...     cello=consort.CompositeMusicSpecifier(
...         primary_music_specifier='one',
...         secondary_music_specifier=consort.MusicSpecifierSequence(
...             music_specifiers=['two', 'three', 'four'],
...             ),
...         )
...     )
\end{lstlisting}
\end{singlespacing}
\end{abjadbookoutput}

\noindent Consider the following string quartet score template, which will be
used with the above music setting. Note the names of the various contexts
defined in it, made visible when formatted as LilyPond syntax, where each
context name is given by a quoted string on the lines beginning with
\texttt{\textbackslash{}context}. The score contains a time signature context,
and four staff groups, one for each instrument in the quartet. These staff
groups then contain two staves, for the left and right hands of each
performer, with each staff containing a single voice:

\begin{comment}
<abjad>
score_template = consort.StringQuartetScoreTemplate()
score = score_template()
print(format(score))
</abjad>
\end{comment}

\begin{abjadbookoutput}
\begin{singlespacing}
\vspace{-0.5\baselineskip}
\begin{lstlisting}
>>> score_template = consort.StringQuartetScoreTemplate()
>>> score = score_template()
>>> print(format(score))
\context Score = "String Quartet Score" <<
    \tag #'time
    \context TimeSignatureContext = "TimeSignatureContext" {
    }
    \tag #'violin-1
    \context StringPerformerGroup = "Violin 1 Performer Group" \with {
        instrumentName = \markup {
            \hcenter-in
                #10
                "Violin 1"
            }
        shortInstrumentName = \markup {
            \hcenter-in
                #10
                "Vln. 1"
            }
    } <<
        \context BowingStaff = "Violin 1 Bowing Staff" {
            \clef "percussion"
            \context Voice = "Violin 1 Bowing Voice" {
            }
        }
        \context FingeringStaff = "Violin 1 Fingering Staff" {
            \clef "treble"
            \context Voice = "Violin 1 Fingering Voice" {
            }
        }
    >>
    \tag #'violin-2
    \context StringPerformerGroup = "Violin 2 Performer Group" \with {
        instrumentName = \markup {
            \hcenter-in
                #10
                "Violin 2"
            }
        shortInstrumentName = \markup {
            \hcenter-in
                #10
                "Vln. 2"
            }
    } <<
        \context BowingStaff = "Violin 2 Bowing Staff" {
            \clef "percussion"
            \context Voice = "Violin 2 Bowing Voice" {
            }
        }
        \context FingeringStaff = "Violin 2 Fingering Staff" {
            \clef "treble"
            \context Voice = "Violin 2 Fingering Voice" {
            }
        }
    >>
    \tag #'viola
    \context StringPerformerGroup = "Viola Performer Group" \with {
        instrumentName = \markup {
            \hcenter-in
                #10
                Viola
            }
        shortInstrumentName = \markup {
            \hcenter-in
                #10
                Va.
            }
    } <<
        \context BowingStaff = "Viola Bowing Staff" {
            \clef "percussion"
            \context Voice = "Viola Bowing Voice" {
            }
        }
        \context FingeringStaff = "Viola Fingering Staff" {
            \clef "alto"
            \context Voice = "Viola Fingering Voice" {
            }
        }
    >>
    \tag #'cello
    \context StringPerformerGroup = "Cello Performer Group" \with {
        instrumentName = \markup {
            \hcenter-in
                #10
                Cello
            }
        shortInstrumentName = \markup {
            \hcenter-in
                #10
                Vc.
            }
    } <<
        \context BowingStaff = "Cello Bowing Staff" {
            \clef "percussion"
            \context Voice = "Cello Bowing Voice" {
            }
        }
        \context FingeringStaff = "Cello Fingering Staff" {
            \clef "bass"
            \context Voice = "Cello Fingering Voice" {
            }
        }
    >>
>>
\end{lstlisting}
\end{singlespacing}
\end{abjadbookoutput}

\noindent Consort's score templates provide abbreviation-to-voice-name mappings
via their \texttt{context\_name\_abbreviations} property. Likewise, they
provide mappings for use with composite music specifiers which map
\enquote{parent} context abbreviations to their relevant child abbreviation
pairs:

\begin{comment}
<abjad>
for abbr, context_name in score_template.context_name_abbreviations.items():
    print(abbr, context_name)

for parent, child_pair in score_template.composite_context_pairs.items():
    print(parent, child_pair)

</abjad>
\end{comment}

\begin{abjadbookoutput}
\begin{singlespacing}
\vspace{-0.5\baselineskip}
\begin{lstlisting}
>>> for abbr, context_name in score_template.context_name_abbreviations.items():
...     print(abbr, context_name)
...
('violin_1', 'Violin 1 Performer Group')
('violin_1_rh', 'Violin 1 Bowing Voice')
('violin_1_lh', 'Violin 1 Fingering Voice')
('violin_2', 'Violin 2 Performer Group')
('violin_2_rh', 'Violin 2 Bowing Voice')
('violin_2_lh', 'Violin 2 Fingering Voice')
('viola', 'Viola Performer Group')
('viola_rh', 'Viola Bowing Voice')
('viola_lh', 'Viola Fingering Voice')
('cello', 'Cello Performer Group')
('cello_rh', 'Cello Bowing Voice')
('cello_lh', 'Cello Fingering Voice')
\end{lstlisting}
\begin{lstlisting}
>>> for parent, child_pair in score_template.composite_context_pairs.items():
...     print(parent, child_pair)
...
('violin_1', ('violin_1_rh', 'violin_1_lh'))
('violin_2', ('violin_2_rh', 'violin_2_lh'))
('viola', ('viola_rh', 'viola_lh'))
('cello', ('cello_rh', 'cello_lh'))
\end{lstlisting}
\end{singlespacing}
\end{abjadbookoutput}

\noindent Resolving the previously defined music specifier against Consort's
string quartet score template dereferences the correct context names, allowing
them to be indexed into any score created by that score template. Note that not
only are the two non-composite music specifiers now associated with voices in
the score, but they have been recreated as music specifier sequences. Likewise,
the composite music specifier has been reconfigured such that it knows the
specific names of the two voices associated with its primary and secondary
music specifiers, themselves recreated as music specifier sequences:

\begin{comment}
<abjad>
result = music_setting.resolve_music_specifiers(score_template)
for context_name, resolved_music_specifier in result.items():
    print('CONTEXT:', context_name)
    print(format(resolved_music_specifier))

</abjad>
\end{comment}

\begin{abjadbookoutput}
\begin{singlespacing}
\vspace{-0.5\baselineskip}
\begin{lstlisting}
>>> result = music_setting.resolve_music_specifiers(score_template)
>>> for context_name, resolved_music_specifier in result.items():
...     print('CONTEXT:', context_name)
...     print(format(resolved_music_specifier))
...
('CONTEXT:', 'Violin 2 Fingering Voice')
consort.tools.MusicSpecifierSequence(
    music_specifiers=datastructuretools.CyclicTuple(
        ['A']
        ),
    )
('CONTEXT:', 'Viola Fingering Voice')
consort.tools.MusicSpecifierSequence(
    music_specifiers=datastructuretools.CyclicTuple(
        ['B', 'C', 'D']
        ),
    )
('CONTEXT:', 'Cello Performer Group')
consort.tools.CompositeMusicSpecifier(
    primary_music_specifier=consort.tools.MusicSpecifierSequence(
        music_specifiers=datastructuretools.CyclicTuple(
            ['one']
            ),
        ),
    primary_voice_name='Cello Bowing Voice',
    secondary_music_specifier=consort.tools.MusicSpecifierSequence(
        music_specifiers=datastructuretools.CyclicTuple(
            ['two', 'three', 'four']
            ),
        ),
    secondary_voice_name='Cello Fingering Voice',
    )
\end{lstlisting}
\end{singlespacing}
\end{abjadbookoutput}

\noindent Resolving a music setting's timespan identifier against the
segment-maker's segment timespan results in one or more target timespans which
can be used as argument to the music setting's timespan-maker:

\begin{comment}
<abjad>
segment_timespan = timespantools.Timespan(0, 8)
show(segment_timespan)
target_timespans = music_setting.resolve_target_timespans(segment_timespan)
show(target_timespans, range_=(0, 8))
</abjad>
\end{comment}

\begin{abjadbookoutput}
\begin{singlespacing}
\vspace{-0.5\baselineskip}
\begin{lstlisting}
>>> segment_timespan = timespantools.Timespan(0, 8)
>>> show(segment_timespan)
\end{lstlisting}
\noindent\includegraphics[max width=\textwidth,]{assets/lilypond-a34adb1cfbda637e38739ddd84494442.pdf}
\begin{lstlisting}
>>> target_timespans = music_setting.resolve_target_timespans(segment_timespan)
>>> show(target_timespans, range_=(0, 8))
\end{lstlisting}
\noindent\includegraphics[max width=\textwidth,]{assets/lilypond-102acd696ed0eb0ce3e2668275527bd3.pdf}
\end{singlespacing}
\end{abjadbookoutput}

\noindent Note that the start and stop offsets of the target timespans resolved
above do not all align at offsets with power-of-two denominators, such as 1/2,
1/4 or 1/8. By specifying a timespan quantization, the target timespans
generated during resolution can be quantized to a grid:

\begin{comment}
<abjad>
target_timespans = music_setting.resolve_target_timespans(
    segment_timespan,
    timespan_quantization=Duration(1, 16),
    )
show(target_timespans, range_=(0, 8))
</abjad>
\end{comment}

\begin{abjadbookoutput}
\begin{singlespacing}
\vspace{-0.5\baselineskip}
\begin{lstlisting}
>>> target_timespans = music_setting.resolve_target_timespans(
...     segment_timespan,
...     timespan_quantization=Duration(1, 16),
...     )
>>> show(target_timespans, range_=(0, 8))
\end{lstlisting}
\noindent\includegraphics[max width=\textwidth,]{assets/lilypond-fdd1e78affa1ee5ba3e3e9b80feee1b2.pdf}
\end{singlespacing}
\end{abjadbookoutput}

\noindent A mask can also be applied to the target timespans:

\begin{comment}
<abjad>
music_setting = new(
    music_setting,
    timespan_identifier__mask_timespan=timespantools.Timespan(
        start_offset=(1, 2),
        stop_offset=7,
        ),
    )
target_timespans = music_setting.resolve_target_timespans(segment_timespan)
show(target_timespans, range_=(0, 8))
</abjad>
\end{comment}

\begin{abjadbookoutput}
\begin{singlespacing}
\vspace{-0.5\baselineskip}
\begin{lstlisting}
>>> music_setting = new(
...     music_setting,
...     timespan_identifier__mask_timespan=timespantools.Timespan(
...         start_offset=(1, 2),
...         stop_offset=7,
...         ),
...     )
>>> target_timespans = music_setting.resolve_target_timespans(segment_timespan)
>>> show(target_timespans, range_=(0, 8))
\end{lstlisting}
\noindent\includegraphics[max width=\textwidth,]{assets/lilypond-435c344c6fe3e22cf86e856a0fe98c16.pdf}
\end{singlespacing}
\end{abjadbookoutput}

\noindent Once resolved, each music setting can call its timespan-maker to
create timespans with the appropriate voice-name-to-music-specifier mapping,
target timespans and layer, adding the contents of the resulting inventory to
the growing maquette of performed and silent timespans produced by previous
music settings. The populating process repeats until no more music settings
remain.

\subsection{Resolving cascading overlap}
\label{ssec:resolving-cascading-overlap}

One of the driving motivations behind Consort is the ability to create musical
textures consisting of multiple overlapping layers, each created by an
independent maker and each with different materials from the other layers,
allowing multiple materials of various provenances to appear in the same
instrumental voice. Still, because acoustic instruments cannot simply
\enquote{create} arbitrary numbers of voices like a synthesizer might, any
overlap in material allocated for a given voice needs to be resolved. Consort
handles resolution of overlap via a tournament, choosing only one material from
a collection of overlapping candidates.

Consider the following three timespan inventories, created with three different
timespan-makers and music-specifier mappings. The first inventory is created
via a flooded timespan-maker, filling the entirety of a target timespan of 0/1
to 19/4 in voices \enquote{Voice 2} and \enquote{Voice 3}. This inventory
behaves like a constant \enquote{background layer} for those two voices:

\begin{comment}
<abjad>
layer_1_timespan_maker = consort.FloodedTimespanMaker()
layer_1_target_timespan = timespantools.Timespan(0, (19, 4))
layer_1_music_specifiers = collections.OrderedDict([
    ('Voice 2', None),
    ('Voice 3', None),
    ])
layer_1 = layer_1_timespan_maker(
    layer=1,
    music_specifiers=layer_1_music_specifiers,
    target_timespan=layer_1_target_timespan,
    )
show(layer_1, key='voice_name', range_=(0, (21, 4)))
</abjad>
\end{comment}

\begin{abjadbookoutput}
\begin{singlespacing}
\vspace{-0.5\baselineskip}
\begin{lstlisting}
>>> layer_1_timespan_maker = consort.FloodedTimespanMaker()
>>> layer_1_target_timespan = timespantools.Timespan(0, (19, 4))
>>> layer_1_music_specifiers = collections.OrderedDict([
...     ('Voice 2', None),
...     ('Voice 3', None),
...     ])
>>> layer_1 = layer_1_timespan_maker(
...     layer=1,
...     music_specifiers=layer_1_music_specifiers,
...     target_timespan=layer_1_target_timespan,
...     )
>>> show(layer_1, key='voice_name', range_=(0, (21, 4)))
\end{lstlisting}
\noindent\includegraphics[max width=\textwidth,]{assets/lilypond-f86c3ef83d7c5ffbef9e47999bb478a9.pdf}
\end{singlespacing}
\end{abjadbookoutput}

\noindent The second timespan inventory is created by a talea timespan-maker.
This inventory covers all four voices -- \enquote{Voice 1}, \enquote{Voice 2},
\enquote{Voice 3} and \enquote{Voice 4} -- with a texture of evenly distributed
phrases and silences. However, unlike the first timespan inventory, this
texture only spans the target timespan of 3/4 to 19/4, guaranteeing that the
background layer for \enquote{Voice 2} and \enquote{Voice 3} is untouched in
the span of 0/1 to 3/4:

\begin{comment}
<abjad>
layer_2_timespan_maker = consort.TaleaTimespanMaker(
    initial_silence_talea=rhythmmakertools.Talea(
        counts=(0, 1, 3),
        denominator=8,
        ),
    playing_groupings=(1, 2),
    playing_talea=rhythmmakertools.Talea(
        counts=(1, 2, 3, 4),
        denominator=4,
        ),
    silence_talea=rhythmmakertools.Talea(
        counts=(5, 3, 1),
        denominator=8,
        ),
    )
layer_2_target_timespan = timespantools.Timespan((3, 4), (19, 4))
layer_2_music_specifiers = collections.OrderedDict([
    ('Voice 1', None),
    ('Voice 2', None),
    ('Voice 3', None),
    ('Voice 4', None),
    ])
layer_2 = layer_2_timespan_maker(
    layer=2,
    music_specifiers=layer_2_music_specifiers,
    target_timespan=layer_2_target_timespan,
    )
show(layer_2, key='voice_name', range_=(0, (21, 4)))
</abjad>
\end{comment}

\begin{abjadbookoutput}
\begin{singlespacing}
\vspace{-0.5\baselineskip}
\begin{lstlisting}
>>> layer_2_timespan_maker = consort.TaleaTimespanMaker(
...     initial_silence_talea=rhythmmakertools.Talea(
...         counts=(0, 1, 3),
...         denominator=8,
...         ),
...     playing_groupings=(1, 2),
...     playing_talea=rhythmmakertools.Talea(
...         counts=(1, 2, 3, 4),
...         denominator=4,
...         ),
...     silence_talea=rhythmmakertools.Talea(
...         counts=(5, 3, 1),
...         denominator=8,
...         ),
...     )
>>> layer_2_target_timespan = timespantools.Timespan((3, 4), (19, 4))
>>> layer_2_music_specifiers = collections.OrderedDict([
...     ('Voice 1', None),
...     ('Voice 2', None),
...     ('Voice 3', None),
...     ('Voice 4', None),
...     ])
>>> layer_2 = layer_2_timespan_maker(
...     layer=2,
...     music_specifiers=layer_2_music_specifiers,
...     target_timespan=layer_2_target_timespan,
...     )
>>> show(layer_2, key='voice_name', range_=(0, (21, 4)))
\end{lstlisting}
\noindent\includegraphics[max width=\textwidth,]{assets/lilypond-e49b088ae657fd66299d7b78da251b1a.pdf}
\end{singlespacing}
\end{abjadbookoutput}

\noindent The third timespan inventory consists of groups of near-simultaneous
attacks in three voices -- \enquote{Voice 1}, \enquote{Voice 3} and
\enquote{Voice 4}. This inventory's talea timespan-maker has padded
1/4-duration silences around the beginning and end of each group, guaranteeing
that any performed timespans with layers lower than 3 will be masked not only
by this layer's performed timespans, but by its silent timespans as well.
Additionally, the third timespan inventory was created with a target timespan
of 6/4 to 21/4. While -- due to the complexities of the talea timespan-maker's
patterns -- the generated timespan texture may not extend all of the way to its
target timespan's stop offset at 21/4, it will certainly not contain any
performed or silent timespans earlier than 6/4, leaving lower layers untouched
from 0/1 to 6/4:

\begin{comment}
<abjad>
layer_3_timespan_maker = consort.TaleaTimespanMaker(
    initial_silence_talea=rhythmmakertools.Talea(
        counts=(0, 0, 0, 1),
        denominator=8,
        ),
    padding=Duration(1, 4),
    playing_talea=rhythmmakertools.Talea(
        counts=(2, 3, 4),
        denominator=8,
        ),
    silence_talea=rhythmmakertools.Talea(
        counts=(6,),
        denominator=4,
        ),
    synchronize_step=True,
    )
layer_3_target_timespan = timespantools.Timespan((6, 4), (21, 4))
layer_3_music_specifiers = collections.OrderedDict([
    ('Voice 1', None),
    ('Voice 3', None),
    ('Voice 4', None),
    ])
layer_3 = layer_3_timespan_maker(
    layer=3,
    music_specifiers=layer_3_music_specifiers,
    target_timespan=layer_3_target_timespan,
    )
show(layer_3, key='voice_name', range_=(0, (21, 4)))
</abjad>
\end{comment}

\begin{abjadbookoutput}
\begin{singlespacing}
\vspace{-0.5\baselineskip}
\begin{lstlisting}
>>> layer_3_timespan_maker = consort.TaleaTimespanMaker(
...     initial_silence_talea=rhythmmakertools.Talea(
...         counts=(0, 0, 0, 1),
...         denominator=8,
...         ),
...     padding=Duration(1, 4),
...     playing_talea=rhythmmakertools.Talea(
...         counts=(2, 3, 4),
...         denominator=8,
...         ),
...     silence_talea=rhythmmakertools.Talea(
...         counts=(6,),
...         denominator=4,
...         ),
...     synchronize_step=True,
...     )
>>> layer_3_target_timespan = timespantools.Timespan((6, 4), (21, 4))
>>> layer_3_music_specifiers = collections.OrderedDict([
...     ('Voice 1', None),
...     ('Voice 3', None),
...     ('Voice 4', None),
...     ])
>>> layer_3 = layer_3_timespan_maker(
...     layer=3,
...     music_specifiers=layer_3_music_specifiers,
...     target_timespan=layer_3_target_timespan,
...     )
>>> show(layer_3, key='voice_name', range_=(0, (21, 4)))
\end{lstlisting}
\noindent\includegraphics[max width=\textwidth,]{assets/lilypond-941b5746ac1f0adcadef1f907d8ffb78.pdf}
\end{singlespacing}
\end{abjadbookoutput}

\noindent Recall from \autoref{sec:timespan-makers} that timespan-makers can
modify a timespan inventory in-place when called, rather than generating a new
one from scratch. The following code -- greatly simplified from Consort's
\texttt{SegmentMaker.populate\_multiplexed\_maquette()} method -- demonstrates
the process of calling multiple timespan-makers with their corresponding target
timespans and voice-name-to-music-specifier mappings to progressively populate
a single timespan inventory in-place. Note the use of two Python builtin
iterators \texttt{zip()} and \texttt{enumerate()}. The \texttt{zip()} iterators
iterates over the iterables with which it was instantiated yielding the first
item of each of its iterables as a tuple, then the second of each of its
iterables, then the third, and so forth. The \texttt{enumerate()} iterator
yields each item of its input iterable paired with that item's index, filling
the role in Python for the verbose \emph{for loop} loop idiom found in many
C-like languages, such as Java or Javascript: \texttt{for(int x = 10; x < 20; x
= x + 1) \{ ... \}}.

\begin{comment}
<abjad>
timespan_makers = (
    layer_1_timespan_maker,
    layer_2_timespan_maker,
    layer_3_timespan_maker,
    )
music_specifiers = (
    layer_1_music_specifiers,
    layer_2_music_specifiers,
    layer_3_music_specifiers,
    )
target_timespans = (
    layer_1_target_timespan,
    layer_2_target_timespan,
    layer_3_target_timespan,
    )
triples = zip(timespan_makers, music_specifiers, target_timespans)
timespan_inventory = timespantools.TimespanInventory()
for layer, triple in enumerate(triples, 1):
    timespan_inventory = triple[0](
        layer=layer,
        music_specifiers=triple[1],
        target_timespan=triple[2],
        timespan_inventory=timespan_inventory,
        )

</abjad>
\end{comment}

\begin{abjadbookoutput}
\begin{singlespacing}
\vspace{-0.5\baselineskip}
\begin{lstlisting}
>>> timespan_makers = (
...     layer_1_timespan_maker,
...     layer_2_timespan_maker,
...     layer_3_timespan_maker,
...     )
>>> music_specifiers = (
...     layer_1_music_specifiers,
...     layer_2_music_specifiers,
...     layer_3_music_specifiers,
...     )
>>> target_timespans = (
...     layer_1_target_timespan,
...     layer_2_target_timespan,
...     layer_3_target_timespan,
...     )
>>> triples = zip(timespan_makers, music_specifiers, target_timespans)
>>> timespan_inventory = timespantools.TimespanInventory()
>>> for layer, triple in enumerate(triples, 1):
...     timespan_inventory = triple[0](
...         layer=layer,
...         music_specifiers=triple[1],
...         target_timespan=triple[2],
...         timespan_inventory=timespan_inventory,
...         )
...
\end{lstlisting}
\end{singlespacing}
\end{abjadbookoutput}

\noindent Recall that the result of this process is still a timespan inventory,
just as described in \autoref{sec:timespan-inventories}. While this document
has taken pains to clarify the internal structure of timespan-maker-generated
timespan inventories by sorting and displaying voice-names in their
illustrations -- just as in the above three timespan inventory illustrations --
that behavior derives from the timespan inventory's illustration protocol
implementation, and does not reflect their actual, flat structure:

\begin{comment}
<abjad>
show(timespan_inventory, range_=(0, (21, 4)))
</abjad>
\end{comment}

\begin{abjadbookoutput}
\begin{singlespacing}
\vspace{-0.5\baselineskip}
\begin{lstlisting}
>>> show(timespan_inventory, range_=(0, (21, 4)))
\end{lstlisting}
\noindent\includegraphics[max width=\textwidth,]{assets/lilypond-39bc7fc954dbcc692b2517c8195b821e.pdf}
\end{singlespacing}
\end{abjadbookoutput}

\noindent Just because timespan inventories are often displayed with
voice-names, and contain performed timespans with voice-name attributes, does
not mean that they are automatically or internally structured that way. Such
sorting requires an additional pass -- demultiplexing. Visualizing the
inventory, exploded by voice-name, then sorted by layer, shows the overlap in
each voice:

\begin{comment}
<abjad>
show(timespan_inventory, key='voice_name', sortkey='layer', range_=(0, (21, 4)))
</abjad>
\end{comment}

\begin{abjadbookoutput}
\begin{singlespacing}
\vspace{-0.5\baselineskip}
\begin{lstlisting}
>>> show(timespan_inventory, key='voice_name', sortkey='layer', range_=(0, (21, 4)))
\end{lstlisting}
\noindent\includegraphics[max width=\textwidth,]{assets/lilypond-7b84f6ec88dc9ff87fba45eab8bd201b.pdf}
\end{singlespacing}
\end{abjadbookoutput}

\noindent In order to resolve cascading overlap, the segment-maker must first
demultiplex the performed and silent timespans in the still-multiplexed
maquette into separate timespan inventories by their voice-name attributes. The
segment-maker further separates each demultiplexed-by-voice-name timespan
inventory into multiple timespan inventories according to their contents' layer
attributes, with the lowest-layered inventory first, and the highest-layered
inventory last. This results in one inventory per-voice, per-timespan-maker
from the maquette population process.\footnote{Why not keep all timespan-maker
output separated from the very beginning? Working with a single multiplexed
timespan inventory for much of the rhythmic interpretation process simplifies
many of the procedures used therein, such as dependent timespan-maker
evaluation, splitting, consolidation, etc. Compared to many of the later
rhythmic operations, such as meter rewriting, multiplexing and demultiplexing
timespan inventories are computationally trivial.} With the maquette fully
demultiplexed, the segment-maker can proceed through each layer in each voice,
from lowest to highest. It progressively subtracts the timespans in each higher
inventory from the lowest inventory, effectively cutting out holes outlining
the shapes of that higher inventory's timespans, then adds that higher
inventory's timespans into the lowest-layered inventory. This process masks
lower-layered timespans with higher ones. The process repeats until no more
timespan inventories remain for that voice, then moves onto the inventories for
the next voice. The resolved, demultiplexed results are finally collected into
a timespan inventory mapping, associating voice names with resolved timespan
inventories, and returned.

\begin{comment}
<abjad>
demultiplexed_maquette = consort.SegmentMaker.resolve_maquette(
    timespan_inventory,
    )
</abjad>
\end{comment}

\begin{abjadbookoutput}
\begin{singlespacing}
\vspace{-0.5\baselineskip}
\begin{lstlisting}
>>> demultiplexed_maquette = consort.SegmentMaker.resolve_maquette(
...     timespan_inventory,
...     )
\end{lstlisting}
\end{singlespacing}
\end{abjadbookoutput}

\noindent After resolution, no overlap remains in the timespans for any voice.
Note too that no silent timespans -- like those created as padding in the third
timespan-maker above -- remain either. Silent timespans act solely as a means
of \enquote{holding space} for a layer, masking but not replacing timespans in
lower layers:

\begin{comment}
<abjad>
show(demultiplexed_maquette, range_=(0, (21, 4)))
</abjad>
\end{comment}

\begin{abjadbookoutput}
\begin{singlespacing}
\vspace{-0.5\baselineskip}
\begin{lstlisting}
>>> show(demultiplexed_maquette, range_=(0, (21, 4)))
\end{lstlisting}
\noindent\includegraphics[max width=\textwidth,]{assets/lilypond-8d7921006d6bce021d9a284d12ed2e90.pdf}
\end{singlespacing}
\end{abjadbookoutput}

\noindent Unlike many of the timespan-handling functions demonstrated in this
chapter as well as in \autoref{ch:time-tools}, \texttt{resolve\_maquette()}
returns a \texttt{TimespanInventoryMapping} rather than a
\texttt{TimespanInventory}. The timespan inventory mapping already
explicitly uses voice-names as keys, obviating the need for a
\texttt{key='voice\_name'} keyword argument pair in the call to
\texttt{show()}.

\subsection{Finding meters, revisited}
\label{ssec:finding-meters-revisited}

Consort's segment-maker implements a variation on the meter-fitting algorithm
described in \autoref{sec:finding-meters}. Each segment-maker may be configured
with an inventory of permitted meters, as well as maximum meter run length, in
order to drive the meter fitting algorithm. When counting offsets,
segment-makers include the offsets found on the performed timespans in their
maquette but discard those from silent timespans, removing any influence from
timespans created solely for silencing other timespans. The start offset of
each performed timespan is weighed twice as much as their stop offset. This
imbalance helps emphasize simultaneous phrase starts across different voices.
Additionally, segment-maker's weight their own desired stop offset at a much
higher value than any count derived from the offsets in their maquette. This
attempts to influence the meter fitting process into selecting a series of
meters which end as close to their desired stop offset as possible. After
fitting meters, the segment-maker caches both the fitted meters and their
boundaries as properties on its instance, affording later retrieval by other
subroutines.

\subsection{Splitting, pruning \& consolidation}
\label{ssec:splitting-pruning-and-consolidation}

Once meters have been fitted against the resolved maquette, the timespans in
the maquette must be split at the measure boundaries outlined by those meters.
Splitting guarantees that no timespans cross any bar-lines and that therefore
no containers generated by those timespans when notating them as score
components cross any bar-lines either. While LilyPond can typeset bar-line
crossing notes, chords and even tuplets, the scores I have composed via Consort
do not currently make use of such constructions. As described in
\autoref{ssec:operations-on-timespans}, operations on timespans which change
offsets -- generating new timespans in the process, rather than modifying the
operated-upon timespan in-place -- preserve their unmodified properties via
templating. Splitting is no exception, and those timespans split maintain their
music specifiers, layer identifiers and voice-names:

\begin{comment}
<abjad>
performed_timespan = consort.PerformedTimespan(
    layer=3,
    start_offset=(1, 2),
    stop_offset=(13, 8),
    voice_name='Percussion Voice',
    )
shards = performed_timespan.split_at_offset((9, 16))
print(format(shards))
</abjad>
\end{comment}

\begin{abjadbookoutput}
\begin{singlespacing}
\vspace{-0.5\baselineskip}
\begin{lstlisting}
>>> performed_timespan = consort.PerformedTimespan(
...     layer=3,
...     start_offset=(1, 2),
...     stop_offset=(13, 8),
...     voice_name='Percussion Voice',
...     )
>>> shards = performed_timespan.split_at_offset((9, 16))
>>> print(format(shards))
timespantools.TimespanInventory(
    [
        consort.tools.PerformedTimespan(
            start_offset=durationtools.Offset(1, 2),
            stop_offset=durationtools.Offset(9, 16),
            layer=3,
            original_stop_offset=durationtools.Offset(13, 8),
            voice_name='Percussion Voice',
            ),
        consort.tools.PerformedTimespan(
            start_offset=durationtools.Offset(9, 16),
            stop_offset=durationtools.Offset(13, 8),
            layer=3,
            original_start_offset=durationtools.Offset(1, 2),
            voice_name='Percussion Voice',
            ),
        ]
    )
\end{lstlisting}
\end{singlespacing}
\end{abjadbookoutput}

\noindent After splitting, the segment-maker prunes timespans considered either
too short or malformed. Performed timespans may be configured with a
\texttt{minimum\_duration} property. Timespan-makers may set this property on
timespans they create when they are themselves configured with a
\texttt{TimespanSpecifier}. Any performed timespan whose actual duration is
less than its minimum duration -- if it has been configured with a minimum
duration -- will be removed from the maquette. Likewise any timespan with a
duration of 0 -- therefore malformed -- will also be removed. While the latter
pruning guarantees correctness of the maquette -- malformed timespans cannot be
rendered as notation at all, and may cause other problems when partitioning due
to ambiguities in their start / stop offset semantics --, the former allows for
a kind of compositional control over the maquette. When notated with certain
rhythm-makers, overly short divisions -- especially those shorter than
1/8-duration -- may give undesirable results. Note that silent timespans have
no configurable minimum duration. Their \texttt{minimum\_duration} always
returns 0. They maintain this dummy property so that the segment-maker's
timespan-pruning algorithms can treat silent and performed timespans
identically.

Next, Consort's segment-maker \emph{consolidates} contiguous performed
timespans with identical music specifiers, caching the durations of the
consolidated timespans in a new timespan's \texttt{divisions} property. Each
new consolidated timespan outlines the start and stop offset of its
consolidated group:

\begin{comment}
<abjad>
timespans = timespantools.TimespanInventory([
    consort.PerformedTimespan(
        start_offset=0,
        stop_offset=10,
        music_specifier='foo',
        ),
    consort.PerformedTimespan(
        start_offset=10,
        stop_offset=20,
        music_specifier='foo',
        ),
    consort.PerformedTimespan(
        start_offset=20,
        stop_offset=25,
        music_specifier='bar',
        ),
    consort.PerformedTimespan(
        start_offset=40,
        stop_offset=50,
        music_specifier='bar',
        ),
    consort.PerformedTimespan(
        start_offset=50,
        stop_offset=58,
        music_specifier='bar',
        ),
    ])
show(timespans)
consolidated_timespans = consort.SegmentMaker.consolidate_timespans(timespans)
show(consolidated_timespans)
</abjad>
\end{comment}

\begin{abjadbookoutput}
\begin{singlespacing}
\vspace{-0.5\baselineskip}
\begin{lstlisting}
>>> timespans = timespantools.TimespanInventory([
...     consort.PerformedTimespan(
...         start_offset=0,
...         stop_offset=10,
...         music_specifier='foo',
...         ),
...     consort.PerformedTimespan(
...         start_offset=10,
...         stop_offset=20,
...         music_specifier='foo',
...         ),
...     consort.PerformedTimespan(
...         start_offset=20,
...         stop_offset=25,
...         music_specifier='bar',
...         ),
...     consort.PerformedTimespan(
...         start_offset=40,
...         stop_offset=50,
...         music_specifier='bar',
...         ),
...     consort.PerformedTimespan(
...         start_offset=50,
...         stop_offset=58,
...         music_specifier='bar',
...         ),
...     ])
>>> show(timespans)
\end{lstlisting}
\noindent\includegraphics[max width=\textwidth,]{assets/lilypond-5f102816b004f8607b165bde7a9a4eb6.pdf}
\begin{lstlisting}
>>> consolidated_timespans = consort.SegmentMaker.consolidate_timespans(timespans)
>>> show(consolidated_timespans)
\end{lstlisting}
\noindent\includegraphics[max width=\textwidth,]{assets/lilypond-dabc429bbedead286bda82e3603f4457.pdf}
\end{singlespacing}
\end{abjadbookoutput}

\noindent Consolidation transforms performed timespans from free-floating cells
in the maquette into components of larger phrases. The cached divisions also
prepare these consolidated timespans for \emph{inscription} by defining the
correct input for a rhythm-maker: a sequence of divisions.

\begin{comment}
<abjad>
print(format(consolidated_timespans))
</abjad>
\end{comment}

\begin{abjadbookoutput}
\begin{singlespacing}
\vspace{-0.5\baselineskip}
\begin{lstlisting}
>>> print(format(consolidated_timespans))
timespantools.TimespanInventory(
    [
        consort.tools.PerformedTimespan(
            start_offset=durationtools.Offset(0, 1),
            stop_offset=durationtools.Offset(20, 1),
            divisions=(
                durationtools.Duration(10, 1),
                durationtools.Duration(10, 1),
                ),
            music_specifier='foo',
            ),
        consort.tools.PerformedTimespan(
            start_offset=durationtools.Offset(20, 1),
            stop_offset=durationtools.Offset(25, 1),
            divisions=(
                durationtools.Duration(5, 1),
                ),
            music_specifier='bar',
            ),
        consort.tools.PerformedTimespan(
            start_offset=durationtools.Offset(40, 1),
            stop_offset=durationtools.Offset(58, 1),
            divisions=(
                durationtools.Duration(10, 1),
                durationtools.Duration(8, 1),
                ),
            music_specifier='bar',
            ),
        ]
    )
\end{lstlisting}
\end{singlespacing}
\end{abjadbookoutput}

\noindent If the music specifier of the consolidated timespan was configured
with a \emph{minimum phrase duration}, and the consolidated timespan falls
under that threshold, it too is discarded.

\subsection{Inscription}
\label{ssec:inscription}

\emph{Inscription} describes the process of generating \emph{music} from a
performed timespan's divisions and rhythm-maker and \emph{inscribing} the
timespan with the result. Consort's segment-maker performs inscription by
iterating over the timespans for each voice in the demultiplexed maquette, in
score order. For each performed timespan encountered, the segment-maker
retrieves that performed timespan's music specifier, and increments that music
specifiers count in a counter. This allows each music specifier to maintain a
seed value while inscribing each performed timespan, even in fragmentary
textures, and to produce continuously varied results from each successive
rhythm-maker belonging to the same music specifier. Recall from
\autoref{ssec:populating-voices} that rhythm-makers can be called not only with
a list of divisions, but also a seed value, rotating the rhythm-makers
sequence-like properties when creating its rhythmic output. All of the handlers
discussed in \autoref{sec:non-rhythmic-interpretation} employ similar -- or
even more complex -- techniques for maintaining state across different phrases
sharing the same music specifier.

The segment-maker also retrieves the performed timespan's division list --
created during consolidation, as described in
\autoref{ssec:splitting-pruning-and-consolidation} -- and its rhythm-maker.
Rhythm-maker retrieval, like seed retrieval, is non-trivial. A performed
timespan's music specifier may not have rhythm-maker defined, or that performed
timespan may not even have a music specifier defined. If a performed timespan
\emph{has} a rhythm-maker defined on its music specifier, the segment-maker
retrieves that. If the timespan has a music specifier, but no defined
rhythm-maker, the segment-maker constructs a note rhythm-maker which ties all
of its divisions together. If the timespan has no music specifier defined at
all, the segment-maker returns a fully-masked note rhythm-maker.

With the performed timespan's seed, rhythm-maker and division list ready, the
segment-maker creates the performed timespan's music. This proceeds almost
identically to the \texttt{make\_music()} function described in
\autoref{ssec:populating-voices}. Consort's segment-maker makes one additional
adjustment on top of that algorithm, replacing trivially-prolated tuplets --
tuplets with ratios of 1:1 -- with unprolated containers.

Next, the segment-maker performs \emph{rest consolidation} on the
newly-generated phrase, grouping all of the phrase's unprolated rests -- those
not appearing in tuplets -- into their own containers, and leaving all other
components -- all notes and chords, and any rests found within a tuplet -- in
their original division within the phrase. Rest consolidation allows the
segment-maker to not only regroup the contents of a phrase into silent and
non-silent segments, but to actually split the performed timespan itself,
creating larger gaps within the maquette, and improving the chances for
notating full-bar rests when finally filling in silences between phrases.

Consider the following phrase-like container -- annotated to show its internal
division structure --, containing divisions in various configurations --
with rests at the beginning, at the end, with prolated rests, no rests at all,
and so forth:

\begin{comment}
<abjad>
parseable = r'''
\new Voice {
    {
        { \time 4/4 r4 c4 }
        { \times 2/3 { c4 r8 } r4 }
        { \time 4/4 r4 c4. r8 }
        { r4 \break }
        { \time 3/4 r4 }
        { c8 c4. }
        \times 2/3 { \time 2/4 r4 c4 c4 }
        { \time 4/4 c4. r8 }
        { r8 c4 r8 }
    }
}
'''
unconsolidated_staff = Staff(parseable, context_name='RhythmicStaff')
consort.annotate(unconsolidated_staff)
show(unconsolidated_staff)
</abjad>
\end{comment}

\begin{abjadbookoutput}
\begin{singlespacing}
\vspace{-0.5\baselineskip}
\begin{lstlisting}
>>> parseable = r'''
... \new Voice {
...     {
...         { \time 4/4 r4 c4 }
...         { \times 2/3 { c4 r8 } r4 }
...         { \time 4/4 r4 c4. r8 }
...         { r4 \break }
...         { \time 3/4 r4 }
...         { c8 c4. }
...         \times 2/3 { \time 2/4 r4 c4 c4 }
...         { \time 4/4 c4. r8 }
...         { r8 c4 r8 }
...     }
... }
... '''
>>> unconsolidated_staff = Staff(parseable, context_name='RhythmicStaff')
>>> consort.annotate(unconsolidated_staff)
>>> show(unconsolidated_staff)
\end{lstlisting}
\noindent\includegraphics[max width=\textwidth,]{assets/lilypond-8a213398dc0a03a0c9fcf4af26637767.pdf}
\end{singlespacing}
\end{abjadbookoutput}

\noindent A clear comparison can be made by duplicating the original phrase,
consolidating its rests, annotating it, and grouping both the original
unconsolidated phrase and the consolidated into a staff group, with the
original above and the altered below:

\begin{comment}
<abjad>
consolidated_staff = Staff(parseable, context_name='RhythmicStaff')
for voice in consolidated_staff:
    for phrase in voice:
        phrase = consort.SegmentMaker.consolidate_rests(phrase)

consort.annotate(consolidated_staff)
staff_group = StaffGroup([unconsolidated_staff, consolidated_staff])
show(staff_group)
</abjad>
\end{comment}

\begin{abjadbookoutput}
\begin{singlespacing}
\vspace{-0.5\baselineskip}
\begin{lstlisting}
>>> consolidated_staff = Staff(parseable, context_name='RhythmicStaff')
>>> for voice in consolidated_staff:
...     for phrase in voice:
...         phrase = consort.SegmentMaker.consolidate_rests(phrase)
...
>>> consort.annotate(consolidated_staff)
>>> staff_group = StaffGroup([unconsolidated_staff, consolidated_staff])
>>> show(staff_group)
\end{lstlisting}
\noindent\includegraphics[max width=\textwidth,]{assets/lilypond-1ef1a4031465a364425aef4e059efda0.pdf}
\end{singlespacing}
\end{abjadbookoutput}

\noindent Note the rests

\begin{comment}
<abjad>
rhythm_maker=rhythmmakertools.NoteRhythmMaker(
    output_masks=[
        rhythmmakertools.BooleanPattern(
            indices=[0],
            period=3,
            ),
        ],
    )
divisions = [Duration(1, 4)] * 7
show(rhythm_maker, divisions=divisions)
</abjad>
\end{comment}

\begin{abjadbookoutput}
\begin{singlespacing}
\vspace{-0.5\baselineskip}
\begin{lstlisting}
>>> rhythm_maker=rhythmmakertools.NoteRhythmMaker(
...     output_masks=[
...         rhythmmakertools.BooleanPattern(
...             indices=[0],
...             period=3,
...             ),
...         ],
...     )
>>> divisions = [Duration(1, 4)] * 7
>>> show(rhythm_maker, divisions=divisions)
\end{lstlisting}
\noindent\includegraphics[max width=\textwidth,]{assets/lilypond-41db2ebb806422340ff52eeaef4859ca.pdf}
\end{singlespacing}
\end{abjadbookoutput}

\noindent This performed timespan

\begin{comment}
<abjad>
timespan = consort.PerformedTimespan(
    divisions=divisions,
    start_offset=0,
    stop_offset=(7, 4),
    music_specifier=consort.MusicSpecifier(
        rhythm_maker=rhythm_maker,
        ),
    )
show(timespan)
</abjad>
\end{comment}

\begin{abjadbookoutput}
\begin{singlespacing}
\vspace{-0.5\baselineskip}
\begin{lstlisting}
>>> timespan = consort.PerformedTimespan(
...     divisions=divisions,
...     start_offset=0,
...     stop_offset=(7, 4),
...     music_specifier=consort.MusicSpecifier(
...         rhythm_maker=rhythm_maker,
...         ),
...     )
>>> show(timespan)
\end{lstlisting}
\noindent\includegraphics[max width=\textwidth,]{assets/lilypond-afc1398492f58ffad86eeaffa0b1a3a3.pdf}
\end{singlespacing}
\end{abjadbookoutput}

\noindent Note the gaps

\begin{comment}
<abjad>
inscribed_timespans = consort.SegmentMaker.inscribe_timespan(timespan)
show(inscribed_timespans, range_=(0, (7, 4)))
print(format(inscribed_timespans))
</abjad>
\end{comment}

\begin{abjadbookoutput}
\begin{singlespacing}
\vspace{-0.5\baselineskip}
\begin{lstlisting}
>>> inscribed_timespans = consort.SegmentMaker.inscribe_timespan(timespan)
>>> show(inscribed_timespans, range_=(0, (7, 4)))
\end{lstlisting}
\noindent\includegraphics[max width=\textwidth,]{assets/lilypond-4626b71066d8fb3d04f25bd125810768.pdf}
\begin{lstlisting}
>>> print(format(inscribed_timespans))
timespantools.TimespanInventory(
    [
        consort.tools.PerformedTimespan(
            start_offset=durationtools.Offset(1, 4),
            stop_offset=durationtools.Offset(3, 4),
            music=scoretools.Container(
                "{   c'4 } {   c'4 }"
                ),
            music_specifier=consort.tools.MusicSpecifier(
                rhythm_maker=rhythmmakertools.NoteRhythmMaker(
                    output_masks=(
                        rhythmmakertools.BooleanPattern(
                            indices=(0,),
                            period=3,
                            ),
                        ),
                    ),
                ),
            original_start_offset=durationtools.Offset(0, 1),
            original_stop_offset=durationtools.Offset(7, 4),
            ),
        consort.tools.PerformedTimespan(
            start_offset=durationtools.Offset(1, 1),
            stop_offset=durationtools.Offset(3, 2),
            music=scoretools.Container(
                "{   c'4 } {   c'4 }"
                ),
            music_specifier=consort.tools.MusicSpecifier(
                rhythm_maker=rhythmmakertools.NoteRhythmMaker(
                    output_masks=(
                        rhythmmakertools.BooleanPattern(
                            indices=(0,),
                            period=3,
                            ),
                        ),
                    ),
                ),
            original_start_offset=durationtools.Offset(0, 1),
            original_stop_offset=durationtools.Offset(7, 4),
            ),
        ]
    )
\end{lstlisting}
\end{singlespacing}
\end{abjadbookoutput}

\noindent The split performed timespans replace the original

Then, MusicSpecifier attachment

\begin{comment}
<abjad>
music = inscribed_timespans[0].music
indicator = inspect_(music).get_indicator(consort.MusicSpecifier)
print(format(indicator))
</abjad>
\end{comment}

\begin{abjadbookoutput}
\begin{singlespacing}
\vspace{-0.5\baselineskip}
\begin{lstlisting}
>>> music = inscribed_timespans[0].music
>>> indicator = inspect_(music).get_indicator(consort.MusicSpecifier)
>>> print(format(indicator))
consort.tools.MusicSpecifier(
    rhythm_maker=rhythmmakertools.NoteRhythmMaker(
        output_masks=(
            rhythmmakertools.BooleanPattern(
                indices=(0,),
                period=3,
                ),
            ),
        ),
    )
\end{lstlisting}
\end{singlespacing}
\end{abjadbookoutput}

\noindent Why attach the music specifier to the phrase container?

\subsection{Meter pruning}
\label{ssec:meter-pruning}

After the timespan pruning outlined in
\autoref{ssec:splitting-pruning-and-consolidation}, and the possibility of gaps
introduced due to rest consolidation as outlined in \autoref{ssec:inscription},
the overall stop offset of the maquette -- not the stop offset derived from the
segment-maker's desired duration -- may have shifted earlier. Depending on the
degree of shift, timespans in the maquette may no longer occur during one or
more of the implicit timespans of the previously fitted meters. Segment-makers
may be configured to discard these silences via their
\texttt{discard\_final\_silence} property, progressively removing meters from
the end of the list of fitted meters until one overlaps at least one performed
timespan in the maquette.

\subsection{Populating dependent timespans}
\label{ssec:populating-dependent-timespans}

The previous few passages, from \autoref{ssec:populating-the-maquette} through
\autoref*{ssec:meter-pruning}, describe the process of populating a
segment-makers's timespan maquette with the products of its *independent* music
settings -- those music settings whose timespan makers are independent, notably
flooded and talea timespan-makers. With the maquette partially populated, those
music settings with dependent timespan-makers -- timespan-makers which generate
timespans based on the contents of a preexisting timespan inventory -- may
finally be called to provide their contributions to the maquette. Dependent
population proceeds almost identically to independent population with a few
notable differences. For one, dependent population dispenses with meter finding
entirely. Consort treats meter as entirely dependent upon independent
timespans, as dependent timespans -- in practice -- are generally used for
keyboard pedaling voices, and should therefore have little bearing on the
overall metrical structure. And while maquette resolving, as described in
\autoref{ssec:resolving-cascading-overlap}, results in a demultiplexed timespan
inventory mapping -- a dictionary of voice-names to timespan inventories --,
independent timespan population completes by \emph{re-multiplexing} that
mapping -- effectively flattening -- back into a single timespan inventory.
This flattening prepares the correct input for dependent timespan population,
as dependent timespans require a single pre-populated timespan inventory as
input. As there are no more passes of timespans to add after dependent timespan
population -- discounting the population of silent timespans, as described in
\autoref{ssec:populating-silent-timespans} -- dependent timespan population
completes with its maquette still demultiplexed.

\subsection{Populating silent timespans}
\label{ssec:populating-silent-timespans}

With the maquette finally populated by inscribed performed timespans, properly
split, resolved and consolidated, the segment-maker can fill in the remaining
gaps between phrases in each voice. This is accomplished by creating
rest-inscribed timespans for each gap. As there are no more layers to add to
the maquette, the segment-maker can populate, split and inscribe timespans for
each of these gaps in a single pass. For each voice in the segment-maker's
still-empty score, the segment-maker retrieves all timespans -- if any --
associated with that voice and subtracts each of them in turn from a single
silent timespan the length of the entire segment, as determined by the first
and last meter boundaries. Any remaining shards from that segment-length silent
timespan represent gaps between phrases in that voice. If the maquette contains
no performed timespans associated with that voice, the segment-length silent
timespan remains unaltered. The segment-maker then splits the remaining silent
timespan shards at every intersecting meter boundary, as described in
\autoref{ssec:splitting-pruning-and-consolidation}, guaranteeing that the
resulting silent timespans do not cross bar-lines. Once split, the
segment-maker partitions the silent timespans, and iterates over the
partitioned groups. For each group of contiguous silent timespans, it generates
a phrase of music containing only rests using a fully-masked note rhythm-maker
-- as described in \autoref{ssec:inscription} --, attaches a dummy
music-specifier to the phrase and instantiates a performed timespan annotated
with that phrase, adding it to the current voice's timespan inventory in the
demultiplex maquette.

Recall the demultiplexed maquette created earlier in
\autoref{ssec:resolving-cascading-overlap}:

\begin{comment}
<abjad>
show(demultiplexed_maquette, range_=(0, 6))
</abjad>
\end{comment}

\begin{abjadbookoutput}
\begin{singlespacing}
\vspace{-0.5\baselineskip}
\begin{lstlisting}
>>> show(demultiplexed_maquette, range_=(0, 6))
\end{lstlisting}
\noindent\includegraphics[max width=\textwidth,]{assets/lilypond-e3cd38c9cc43be27a4d132c287a15a2a.pdf}
\end{singlespacing}
\end{abjadbookoutput}

\noindent We can simulate the process of populating silent timespans in the
following code, creating silent timespans -- purely for visualization purposes
-- rather than the rest-inscribed timespans actually created during silent
timespan population:

\begin{comment}
<abjad>
segment_duration = Duration(24, 4)
for voice_name, timespan_inventory in demultiplexed_maquette.items():
    silent_timespans = timespantools.TimespanInventory([
        consort.SilentTimespan(
            start_offset=0,
            stop_offset=segment_duration,
            voice_name=voice_name,
            ),
        ])
    for timespan in timespan_inventory:
        silent_timespans -= timespan
    timespan_inventory.extend(silent_timespans)
    timespan_inventory.sort()

show(demultiplexed_maquette, range_=(0, 6))
</abjad>
\end{comment}

\begin{abjadbookoutput}
\begin{singlespacing}
\vspace{-0.5\baselineskip}
\begin{lstlisting}
>>> segment_duration = Duration(24, 4)
>>> for voice_name, timespan_inventory in demultiplexed_maquette.items():
...     silent_timespans = timespantools.TimespanInventory([
...         consort.SilentTimespan(
...             start_offset=0,
...             stop_offset=segment_duration,
...             voice_name=voice_name,
...             ),
...         ])
...     for timespan in timespan_inventory:
...         silent_timespans -= timespan
...     timespan_inventory.extend(silent_timespans)
...     timespan_inventory.sort()
...
>>> show(demultiplexed_maquette, range_=(0, 6))
\end{lstlisting}
\noindent\includegraphics[max width=\textwidth,]{assets/lilypond-c49d2c79b38a6d5fdf78db3846702135.pdf}
\end{singlespacing}
\end{abjadbookoutput}

\noindent The above assumes a segment duration of 6 instead of the maquette's
initial duration of 21/4, and therefore pads out the end of each timespan
inventory to that stop offset with silence.

\subsection{Rewriting meters, revisited}
\label{ssec:rewriting-meters-revisited}

Once its maquette is completely populated, Consort's segment-maker performs
meter rewriting. This proceeds in a more elaborate manner than the meter
rewriting process as outlined in \autoref{sec:rewriting-meters} and
\autoref{ssec:rewriting-meters}, and involves a number of notable differences.

For reasons of computational efficiency, Consort rewrites the meters in
each phrase of music annotating each performed timespan in the maquette before
those phrases have even been inserted into the segment-maker's score. Meter
rewriting involves potentially many alterations to the contents of containers
due to fusing and splitting, as well as many duration and offset lookups.
Anytime a component is replaced or its duration changed, the cached offsets of
components located in the score tree after the changed component as well as
the durations of its parents are invalidated. They must be recomputed and
re-cached on the next offset lookup performed on any component the score tree.
Delaying inserting each inscribed timespan's music into the segment-maker's
score guarantees that that music's score depth remains shallow, and therefore
limits the complexity of offset calculation during rewriting.

When beginning the meter rewriting process, the segment-maker converts its
fitted meters into a timespan collection -- Consort's optimized timespan
inventory class -- containing timespans annotated with those fitted meters, one
per timespan.

\begin{comment}
<abjad>
meters = metertools.MeterInventory([(3, 4), (2, 4), (6, 8), (5, 16)])
meter_timespans = consort.SegmentMaker.meters_to_timespans(meters)
print(format(meter_timespans))
</abjad>
\end{comment}

\begin{abjadbookoutput}
\begin{singlespacing}
\vspace{-0.5\baselineskip}
\begin{lstlisting}
>>> meters = metertools.MeterInventory([(3, 4), (2, 4), (6, 8), (5, 16)])
>>> meter_timespans = consort.SegmentMaker.meters_to_timespans(meters)
>>> print(format(meter_timespans))
consort.tools.TimespanCollection(
    [
        timespantools.AnnotatedTimespan(
            start_offset=durationtools.Offset(0, 1),
            stop_offset=durationtools.Offset(3, 4),
            annotation=metertools.Meter(
                '(3/4 (1/4 1/4 1/4))'
                ),
            ),
        timespantools.AnnotatedTimespan(
            start_offset=durationtools.Offset(3, 4),
            stop_offset=durationtools.Offset(5, 4),
            annotation=metertools.Meter(
                '(2/4 (1/4 1/4))'
                ),
            ),
        timespantools.AnnotatedTimespan(
            start_offset=durationtools.Offset(5, 4),
            stop_offset=durationtools.Offset(2, 1),
            annotation=metertools.Meter(
                '(6/8 ((3/8 (1/8 1/8 1/8)) (3/8 (1/8 1/8 1/8))))'
                ),
            ),
        timespantools.AnnotatedTimespan(
            start_offset=durationtools.Offset(2, 1),
            stop_offset=durationtools.Offset(37, 16),
            annotation=metertools.Meter(
                '(5/16 (1/16 1/16 1/16 1/16 1/16))'
                ),
            ),
        ]
    )
\end{lstlisting}
\end{singlespacing}
\end{abjadbookoutput}

\noindent Representing meters as timespans provides two important benefits.
First, meters intersecting a given division within a phrase can be efficiently
located using the search methods implemented on \texttt{TimespanCollection}.
Second, the time relation methods implemented on \texttt{Timespan} can be used
to test if a given division's timespan is congruent -- that is, possesses an
identical start and stop offset -- to a meter's timespan. Divisions containing
solely rests which are also congruent to a meter timespan can be rewritten with
full-bar rests.

The segment-maker then proceeds through each demultiplexed timespan inventory
in the maquette, iterating over each timespan, then over each division in that
performed timespan's music. Timespans whose rhythm-maker forbids meter
rewriting -- via the \texttt{forbid\_meter\_rewriting} flag on an optional
\texttt{DurationSpellingSpecifier} -- are skipped over.\footnote{It may be
undesirable to rewrite a rhythm's meter in certain situations, particularly if
a composer is trying to avoid the introduction of ties or dots for whatever
reason.} In order to determine which meter governs a division, that division's
timespan must be retrieved and then translated before it can be used to search
the inventory of meter timespans. Because each phrase of music annotating each
performed timespan has not yet been inserted into the score, they all express
their start offset as 0. Likewise, each phrase's child divisions express their
start offsets relative to their parent phrase's 0 start offset. Translating
each division's timespan relative to the start offset of the performed timespan
annotated by that division's phrase provides a useful search target for the
meter timespan inventory. The translated division timespan represents the
timespan that division \emph{would} occupy if its phrase, and all other
phrases, had already been inserted into the appropriate voice in the score.
Intersecting meters can then be found through a simple search and retranslated
relative to the current performed timespan's start offset, giving their
appropriate location within the not-yet-inserted phrase. The following
operations outline the translation and search process:

\begin{comment}
<abjad>
inscribed_timespan = consort.PerformedTimespan(
    start_offset=(5, 4),
    stop_offset=(9, 5),
    music=Container("{ c'4 }{ c'2 }{ c'4 }"),
    )
division = inscribed_timespan.music[1]
division_timespan = inspect_(division).get_timespan()
print(format(division_timespan))
translation = inscribed_timespan.start_offset
division_timespan = division_timespan.translate(translation)
print(format(division_timespan))
meter_timespan = meter_timespans.find_timespans_intersecting_timespan(
    division_timespan)[0]
print(format(meter_timespan))
translation = -1 * division_timespan.start_offset
meter_timespan = meter_timespan.translate(translation)
print(format(meter_timespan))
</abjad>
\end{comment}

\begin{abjadbookoutput}
\begin{singlespacing}
\vspace{-0.5\baselineskip}
\begin{lstlisting}
>>> inscribed_timespan = consort.PerformedTimespan(
...     start_offset=(5, 4),
...     stop_offset=(9, 5),
...     music=Container("{ c'4 }{ c'2 }{ c'4 }"),
...     )
>>> division = inscribed_timespan.music[1]
>>> division_timespan = inspect_(division).get_timespan()
>>> print(format(division_timespan))
timespantools.Timespan(
    start_offset=durationtools.Offset(1, 4),
    stop_offset=durationtools.Offset(3, 4),
    )
\end{lstlisting}
\begin{lstlisting}
>>> translation = inscribed_timespan.start_offset
>>> division_timespan = division_timespan.translate(translation)
>>> print(format(division_timespan))
timespantools.Timespan(
    start_offset=durationtools.Offset(3, 2),
    stop_offset=durationtools.Offset(2, 1),
    )
\end{lstlisting}
\begin{lstlisting}
>>> meter_timespan = meter_timespans.find_timespans_intersecting_timespan(
...     division_timespan)[0]
>>> print(format(meter_timespan))
timespantools.AnnotatedTimespan(
    start_offset=durationtools.Offset(5, 4),
    stop_offset=durationtools.Offset(2, 1),
    annotation=metertools.Meter(
        '(6/8 ((3/8 (1/8 1/8 1/8)) (3/8 (1/8 1/8 1/8))))'
        ),
    )
\end{lstlisting}
\begin{lstlisting}
>>> translation = -1 * division_timespan.start_offset
>>> meter_timespan = meter_timespan.translate(translation)
>>> print(format(meter_timespan))
timespantools.AnnotatedTimespan(
    start_offset=durationtools.Offset(-1, 4),
    stop_offset=durationtools.Offset(1, 2),
    annotation=metertools.Meter(
        '(6/8 ((3/8 (1/8 1/8 1/8)) (3/8 (1/8 1/8 1/8))))'
        ),
    )
\end{lstlisting}
\end{singlespacing}
\end{abjadbookoutput}

\noindent With the appropriate meters selected, rewriting continues very much
as described in \autoref{ssec:rewriting-meters}. Tuplets are rewritten solely
with respect for the pre-prolated contents durations, and unprolated containers
are rewritten with respect for their intersecting meter, with an initial
offsets applied to the meter rewriting process if they happen to start later
than their meter. Additionally, Consort's meter rewriting tests silent meters
-- those whose leaves consist entirely of rests -- for congruency with the
current meter. Any division consisting solely of rests which also begins and
ends at the start and stop offsets of a meter's timespan can be rewritten
instead as a single full-bar rest. The segment-maker also attaches a
\texttt{StaffLinesSpanner} to the silent division, which collapses the staff
down to a single line, giving the score a fragmented appearance. Finally,
Consort's segment-maker performs a logical-tie cleanup pass, fusing all
2-length logical ties consisting of matched pairs of 1/16 or 1/32 notes. This
takes care of some possible artifacts of heavily subdivided meter-rewriting,
and makes the final rhythmic output generally more readable.

\subsection{Populating the score}
\label{ssec:populating-the-score}

After meter-rewriting, the segment-maker can finally populate its score. To do
so, it iterates through its timespan maquette and still-unpopulated score in
parallel, extracting the inscribed music from each performed timespan in the
maquette and inserting those phrases into the score in the appropriate voice.
With the segment-maker's score populated, rhythmic interpretation ends and
non-rhythmic interpretation may begin.

\section{Non-rhythmic interpretation}
\label{sec:non-rhythmic-interpretation}

\subsection{Score traversal}

\begin{markdown}
-   AttackPointSignature
    -   division_index
    -   phrase_index
    -   logical_tie_index
    -   phrase_position (scaled position within the phrase)
    -   division_position (scaled position within the division)
    -   segment_position (scaled position within the entire segment)
    -   is_first_of_division
    -   is_first_of_phrase
-   Elucidate voice / phrase / division structure
-   SegmentMaker.logical_tie_to_division()
-   SegmentMaker.logical_tie_to_phrase()
-   SegmentMaker.logical_tie_to_voice()
\end{markdown}

The various handlers which iterate through the score during non-rhythmic
interpretation use one of two techniques.

*Voice-wise* iterations iterates through all voice contexts in the score, then
iterates through the top-level containers in those voices. These top-level
containers are the same containers reference by each performed timespan's
\texttt{music} property, and represent each phrase in the maquette.

*Attack-point* iteration iterates through all logical ties in the score in
*time order* according to their start offset in the score, regardless of their
vertical position within the score. Logical ties with identical start offsets
-- those appearing at simultaneities across voices -- are then sorted by their
*score order*. This results in an iteration which moves first forward in time
and the top-of-score to bottom.

\begin{comment}
<abjad>[stylesheet=../consort.ily]
music_specifier = consort.MusicSpecifier(
    attachment_handler=consort.AttachmentHandler(),
    rhythm_maker=rhythmmakertools.TaleaRhythmMaker(
        extra_counts_per_division=(0, 1),
        talea=rhythmmakertools.Talea([2, 3, 2, 4], 16),
        ),
    )
timespan_maker = consort.TaleaTimespanMaker(
    initial_silence_talea=rhythmmakertools.Talea([0, 1], 4),
    playing_groupings=(1, 2, 2, 1, 2),
    playing_talea=rhythmmakertools.Talea([2, 3], 8),
    silence_talea=rhythmmakertools.Talea([1, 2, 3, 4], 8),
    )
music_setting = consort.MusicSetting(
    timespan_maker=timespan_maker,
    v1=music_specifier,
    v2=music_specifier,
    )
segment_maker = consort.SegmentMaker(
    desired_duration_in_seconds=8,
    discard_final_silence=True,
    permitted_time_signatures=[(2, 4), (5, 16), (3, 4)],
    score_template=templatetools.GroupedRhythmicStavesScoreTemplate(
        staff_count=2,
        with_clefs=True,
        ),
    settings=[music_setting],
    tempo=indicatortools.Tempo((1, 4), 72),
    )
illustration = segment_maker(annotate=True, verbose=False)
show(illustration)
</abjad>
\end{comment}

\begin{abjadbookoutput}
\begin{singlespacing}
\vspace{-0.5\baselineskip}
\begin{lstlisting}
>>> music_specifier = consort.MusicSpecifier(
...     attachment_handler=consort.AttachmentHandler(),
...     rhythm_maker=rhythmmakertools.TaleaRhythmMaker(
...         extra_counts_per_division=(0, 1),
...         talea=rhythmmakertools.Talea([2, 3, 2, 4], 16),
...         ),
...     )
>>> timespan_maker = consort.TaleaTimespanMaker(
...     initial_silence_talea=rhythmmakertools.Talea([0, 1], 4),
...     playing_groupings=(1, 2, 2, 1, 2),
...     playing_talea=rhythmmakertools.Talea([2, 3], 8),
...     silence_talea=rhythmmakertools.Talea([1, 2, 3, 4], 8),
...     )
>>> music_setting = consort.MusicSetting(
...     timespan_maker=timespan_maker,
...     v1=music_specifier,
...     v2=music_specifier,
...     )
>>> segment_maker = consort.SegmentMaker(
...     desired_duration_in_seconds=8,
...     discard_final_silence=True,
...     permitted_time_signatures=[(2, 4), (5, 16), (3, 4)],
...     score_template=templatetools.GroupedRhythmicStavesScoreTemplate(
...         staff_count=2,
...         with_clefs=True,
...         ),
...     settings=[music_setting],
...     tempo=indicatortools.Tempo((1, 4), 72),
...     )
>>> illustration = segment_maker(annotate=True, verbose=False)
>>> show(illustration)
\end{lstlisting}
\noindent\includegraphics[max width=\textwidth,]{assets/lilypond-140281b0d71dbf5aae188d9043c542cf.pdf}
\end{singlespacing}
\end{abjadbookoutput}

\begin{comment}
<abjad>[stylesheet=../consort.ily]
for index, key_value_pair in enumerate(segment_maker.attack_point_map.items()):
    logical_tie, attack_point_signature = key_value_pair
    markup = Markup(index, Up)
    markup = markup.smaller().pad_around(0.5).box()
    attach(markup, logical_tie.head)

show(illustration)
</abjad>
\end{comment}

\begin{abjadbookoutput}
\begin{singlespacing}
\vspace{-0.5\baselineskip}
\begin{lstlisting}
>>> for index, key_value_pair in enumerate(segment_maker.attack_point_map.items()):
...     logical_tie, attack_point_signature = key_value_pair
...     markup = Markup(index, Up)
...     markup = markup.smaller().pad_around(0.5).box()
...     attach(markup, logical_tie.head)
...
>>> show(illustration)
\end{lstlisting}
\noindent\includegraphics[max width=\textwidth,]{assets/lilypond-c8b990269b792e5f5a1bb7a581945dcf.pdf}
\end{singlespacing}
\end{abjadbookoutput}

\begin{comment}
<abjad>[stylesheet=../consort.ily]
for index, key_value_pair in enumerate(segment_maker.attack_point_map.items()):
    logical_tie, attack_point_signature = key_value_pair
    for markup in inspect_(logical_tie.head).get_markup():
        detached = detach(markup, logical_tie.head)
    string = '{}:{}'.format(
        attack_point_signature.division_index,
        attack_point_signature.logical_tie_index,
        )
    markup = Markup(string, Up)
    markup = markup.smaller().pad_around(0.5).box()
    attach(markup, logical_tie.head)

show(illustration)
</abjad>
\end{comment}

\begin{abjadbookoutput}
\begin{singlespacing}
\vspace{-0.5\baselineskip}
\begin{lstlisting}
>>> for index, key_value_pair in enumerate(segment_maker.attack_point_map.items()):
...     logical_tie, attack_point_signature = key_value_pair
...     for markup in inspect_(logical_tie.head).get_markup():
...         detached = detach(markup, logical_tie.head)
...     string = '{}:{}'.format(
...         attack_point_signature.division_index,
...         attack_point_signature.logical_tie_index,
...         )
...     markup = Markup(string, Up)
...     markup = markup.smaller().pad_around(0.5).box()
...     attach(markup, logical_tie.head)
...
>>> show(illustration)
\end{lstlisting}
\noindent\includegraphics[max width=\textwidth,]{assets/lilypond-20ecb4d9ff876902b4f8129814dfda62.pdf}
\end{singlespacing}
\end{abjadbookoutput}

\begin{comment}
<abjad>[stylesheet=../consort.ily]
for index, key_value_pair in enumerate(segment_maker.attack_point_map.items()):
    logical_tie, attack_point_signature = key_value_pair
    for markup in inspect_(logical_tie.head).get_markup():
        detached = detach(markup, logical_tie.head)
    phrase_position = attack_point_signature.phrase_position
    markup = Markup.fraction(phrase_position)
    markup = Markup(markup, Up)
    markup = markup.smaller().pad_around(0.5).box()
    attach(markup, logical_tie.head)

show(illustration)
</abjad>
\end{comment}

\begin{abjadbookoutput}
\begin{singlespacing}
\vspace{-0.5\baselineskip}
\begin{lstlisting}
>>> for index, key_value_pair in enumerate(segment_maker.attack_point_map.items()):
...     logical_tie, attack_point_signature = key_value_pair
...     for markup in inspect_(logical_tie.head).get_markup():
...         detached = detach(markup, logical_tie.head)
...     phrase_position = attack_point_signature.phrase_position
...     markup = Markup.fraction(phrase_position)
...     markup = Markup(markup, Up)
...     markup = markup.smaller().pad_around(0.5).box()
...     attach(markup, logical_tie.head)
...
>>> show(illustration)
\end{lstlisting}
\noindent\includegraphics[max width=\textwidth,]{assets/lilypond-8a7f2e5f06a7847e71709bc13d2bd0c7.pdf}
\end{singlespacing}
\end{abjadbookoutput}

\begin{comment}
<abjad>[stylesheet=../consort.ily]
for index, key_value_pair in enumerate(segment_maker.attack_point_map.items()):
    logical_tie, attack_point_signature = key_value_pair
    for markup in inspect_(logical_tie.head).get_markup():
        detached = detach(markup, logical_tie.head)
    segment_position = attack_point_signature.segment_position
    markup = Markup.fraction(segment_position)
    markup = Markup(markup, Up)
    markup = markup.smaller().pad_around(0.5).box()
    attach(markup, logical_tie.head)

show(illustration)
</abjad>
\end{comment}

\begin{abjadbookoutput}
\begin{singlespacing}
\vspace{-0.5\baselineskip}
\begin{lstlisting}
>>> for index, key_value_pair in enumerate(segment_maker.attack_point_map.items()):
...     logical_tie, attack_point_signature = key_value_pair
...     for markup in inspect_(logical_tie.head).get_markup():
...         detached = detach(markup, logical_tie.head)
...     segment_position = attack_point_signature.segment_position
...     markup = Markup.fraction(segment_position)
...     markup = Markup(markup, Up)
...     markup = markup.smaller().pad_around(0.5).box()
...     attach(markup, logical_tie.head)
...
>>> show(illustration)
\end{lstlisting}
\noindent\includegraphics[max width=\textwidth,]{assets/lilypond-80f54cbdf2aeef9dd896ffa785f19332.pdf}
\end{singlespacing}
\end{abjadbookoutput}

Many of the handlers used during non-rhythmic interpretation rely on
information about each pitched logical tie in the score.

Including the position of the logical tie's head in the score regardless of
voice -- its first leaf, and effectively its attack point -- as well as that
head's index both within the division it occurs within, as well as which
division within its phrase it occurs within.

\subsection{Grace-handlers}
\label{ssec:grace-handlers}

Grace handlers attach *grace containers* to logical ties within a score in a
patterned way.

Abjad implements grace notes as normal leaves -- notes, chords and rests --
within special components which act both as Abjad \texttt{Container} classes as
well as attachable indicators. That is to say, grace notes must be placed
within one of these grace containers which is then attached to another leaf in
the score much like any other indicator, such as dynamics or articulations.

Grace-handlers traverse the score by logical tie in time-order, by iterating
over the previously cached ordered dictionary of attack-points, generated at
the end of rhythmic interpretation.

Discuss sub-optimal grace-note spacing in proportional notation?

\begin{comment}
<abjad>[stylesheet=../consort.ily]
music_specifier = consort.MusicSpecifier(
    grace_handler=consort.GraceHandler(
        counts=(1, 2, 0, 0, 0),
        ),
    rhythm_maker=rhythmmakertools.TaleaRhythmMaker(
        extra_counts_per_division=(0, 1),
        talea=rhythmmakertools.Talea([1, 2, 3, 1, 4], 16),
        ),
    )
timespan_maker = consort.TaleaTimespanMaker(
    initial_silence_talea=rhythmmakertools.Talea([0, 1], 4),
    playing_groupings=(1, 2, 2),
    playing_talea=rhythmmakertools.Talea([2, 3], 8),
    silence_talea=rhythmmakertools.Talea([1, 2, 3, 4], 8),
    )
music_setting = consort.MusicSetting(
    timespan_maker=timespan_maker,
    v1=music_specifier,
    v2=music_specifier,
    )
segment_maker = consort.SegmentMaker(
    desired_duration_in_seconds=8,
    discard_final_silence=True,
    permitted_time_signatures=[(2, 4), (5, 16), (3, 4)],
    score_template=templatetools.GroupedRhythmicStavesScoreTemplate(
        staff_count=2,
        with_clefs=True,
        ),
    settings=[music_setting],
    tempo=indicatortools.Tempo((1, 4), 72),
    )
show(segment_maker, verbose=False)
</abjad>
\end{comment}

\begin{abjadbookoutput}
\begin{singlespacing}
\vspace{-0.5\baselineskip}
\begin{lstlisting}
>>> music_specifier = consort.MusicSpecifier(
...     grace_handler=consort.GraceHandler(
...         counts=(1, 2, 0, 0, 0),
...         ),
...     rhythm_maker=rhythmmakertools.TaleaRhythmMaker(
...         extra_counts_per_division=(0, 1),
...         talea=rhythmmakertools.Talea([1, 2, 3, 1, 4], 16),
...         ),
...     )
>>> timespan_maker = consort.TaleaTimespanMaker(
...     initial_silence_talea=rhythmmakertools.Talea([0, 1], 4),
...     playing_groupings=(1, 2, 2),
...     playing_talea=rhythmmakertools.Talea([2, 3], 8),
...     silence_talea=rhythmmakertools.Talea([1, 2, 3, 4], 8),
...     )
>>> music_setting = consort.MusicSetting(
...     timespan_maker=timespan_maker,
...     v1=music_specifier,
...     v2=music_specifier,
...     )
>>> segment_maker = consort.SegmentMaker(
...     desired_duration_in_seconds=8,
...     discard_final_silence=True,
...     permitted_time_signatures=[(2, 4), (5, 16), (3, 4)],
...     score_template=templatetools.GroupedRhythmicStavesScoreTemplate(
...         staff_count=2,
...         with_clefs=True,
...         ),
...     settings=[music_setting],
...     tempo=indicatortools.Tempo((1, 4), 72),
...     )
>>> show(segment_maker, verbose=False)
\end{lstlisting}
\noindent\includegraphics[max width=\textwidth,]{assets/lilypond-fcdf7d5554035d7afe8bf1fb99bbe06f.pdf}
\end{singlespacing}
\end{abjadbookoutput}

\subsection{Pitch-handlers}
\label{ssec:pitch-handlers}

Pitch-handlers manage the process of applying pitch material to logical ties
within a score in a patterned way.

Pitch material may be semantic or non-semantic.

\begin{markdown}
-   Applies pitches to logical ties in a patterned way.
-   Applies pitches to grace notes associated with a logical tie.
-   Applies logical-tie-expressions which can convert logical ties from
    notes into chords, key-clusters or harmonics.
-   Processes the score time-wise by logical tie.
    -   The goal is to limit pitch class repetition both vertically and
        horizontally, but *only* with regard to phrases in the score scoped
        by each music specifier. Other phrases are not considered.
-   Application rate: by logical tie, division, phrase
    -   This requires the SeedSession class for keeping track of many
        seeds, each advancing at a potentially different rate.
    -   This also requires the AttackPointSignature class, which caches
        information about each logical tie's position in its parent
        division, phrase and overall segment.
-   MusicSpecifier: pitches are non-semantic (is this even used?)
-   Maps different patterns of pitches and different patterns of operations
    across the timeline.
    -   PitchHandler: `get_pitch_choice_timespans()`
    -   Demonstrate.
-   Can act on absolute pitches, or registered pitch classes.
-   Other formulations are possible: selecting from vertical sonorities
    based on register curves. (This is not currently implemented, but maybe
    for Ersilia.)
-   Demonstrate simple PitchHandler examples.
-   Additionally:
    -   SeedSession
    -   Pitch operations
    -   Logical tie expressions
    -   Pitch application rate
    -   Pitch specifier
    -   Grace expressions
\end{markdown}

\begin{comment}
<abjad>[stylesheet=../consort.ily]
segment_maker = consort.SegmentMaker(
    desired_duration_in_seconds=9,
    #omit_stylesheets=True,
    permitted_time_signatures=[(3, 4)],
    score_template=templatetools.GroupedStavesScoreTemplate(
        staff_count=2,
        ),
    tempo=indicatortools.Tempo((1, 4), 60),
    )
music_specifier = consort.MusicSpecifier(
    rhythm_maker=rhythmmakertools.TaleaRhythmMaker(
        talea=rhythmmakertools.Talea([1], 16),
        ),
    )
timespan_maker = consort.TaleaTimespanMaker(
    initial_silence_talea=rhythmmakertools.Talea([0, 1], 4),
    playing_talea=rhythmmakertools.Talea([1], 8),
    playing_groupings=[3],
    silence_talea=rhythmmakertools.Talea([1], 8),
    )
segment_maker.add_setting(
    timespan_maker=timespan_maker,
    v1=music_specifier,
    v2=music_specifier,
    )
show(segment_maker, verbose=False)
</abjad>
\end{comment}

\begin{abjadbookoutput}
\begin{singlespacing}
\vspace{-0.5\baselineskip}
\begin{lstlisting}
>>> segment_maker = consort.SegmentMaker(
...     desired_duration_in_seconds=9,
...     #omit_stylesheets=True,
...     permitted_time_signatures=[(3, 4)],
...     score_template=templatetools.GroupedStavesScoreTemplate(
...         staff_count=2,
...         ),
...     tempo=indicatortools.Tempo((1, 4), 60),
...     )
>>> music_specifier = consort.MusicSpecifier(
...     rhythm_maker=rhythmmakertools.TaleaRhythmMaker(
...         talea=rhythmmakertools.Talea([1], 16),
...         ),
...     )
>>> timespan_maker = consort.TaleaTimespanMaker(
...     initial_silence_talea=rhythmmakertools.Talea([0, 1], 4),
...     playing_talea=rhythmmakertools.Talea([1], 8),
...     playing_groupings=[3],
...     silence_talea=rhythmmakertools.Talea([1], 8),
...     )
>>> segment_maker.add_setting(
...     timespan_maker=timespan_maker,
...     v1=music_specifier,
...     v2=music_specifier,
...     )
>>> show(segment_maker, verbose=False)
\end{lstlisting}
\noindent\includegraphics[max width=\textwidth,]{assets/lilypond-394f358f60132ecf989d89a8a9e62012.pdf}
\end{singlespacing}
\end{abjadbookoutput}

\begin{comment}
<abjad>[stylesheet=../consort.ily]
music_specifier = new(
    music_specifier,
    pitch_handler=consort.AbsolutePitchHandler(
        pitch_specifier="c' d' e' f' g' a' b' c''",
        ),
    )
music_setting = consort.MusicSetting(
    timespan_maker=timespan_maker,
    v1=music_specifier,
    v2=music_specifier,
    )
segment_maker = new(segment_maker, settings=[music_setting])
show(segment_maker, verbose=False)
</abjad>
\end{comment}

\begin{abjadbookoutput}
\begin{singlespacing}
\vspace{-0.5\baselineskip}
\begin{lstlisting}
>>> music_specifier = new(
...     music_specifier,
...     pitch_handler=consort.AbsolutePitchHandler(
...         pitch_specifier="c' d' e' f' g' a' b' c''",
...         ),
...     )
>>> music_setting = consort.MusicSetting(
...     timespan_maker=timespan_maker,
...     v1=music_specifier,
...     v2=music_specifier,
...     )
>>> segment_maker = new(segment_maker, settings=[music_setting])
>>> show(segment_maker, verbose=False)
\end{lstlisting}
\noindent\includegraphics[max width=\textwidth,]{assets/lilypond-104b2b0b061ac0a12d87191eadfae379.pdf}
\end{singlespacing}
\end{abjadbookoutput}

\begin{comment}
<abjad>[stylesheet=../consort.ily]
music_specifier = new(
    music_specifier,
    rhythm_maker__talea__counts=[1, 2, 3, 4],
    )
music_setting = consort.MusicSetting(
    timespan_maker=timespan_maker,
    v1=music_specifier,
    v2=music_specifier,
    )
segment_maker = new(segment_maker, settings=[music_setting])
show(segment_maker, verbose=False)
</abjad>
\end{comment}

\begin{abjadbookoutput}
\begin{singlespacing}
\vspace{-0.5\baselineskip}
\begin{lstlisting}
>>> music_specifier = new(
...     music_specifier,
...     rhythm_maker__talea__counts=[1, 2, 3, 4],
...     )
>>> music_setting = consort.MusicSetting(
...     timespan_maker=timespan_maker,
...     v1=music_specifier,
...     v2=music_specifier,
...     )
>>> segment_maker = new(segment_maker, settings=[music_setting])
>>> show(segment_maker, verbose=False)
\end{lstlisting}
\noindent\includegraphics[max width=\textwidth,]{assets/lilypond-1548051ace14e831b1685f04ffc9bf04.pdf}
\end{singlespacing}
\end{abjadbookoutput}

\begin{comment}
<abjad>[stylesheet=../consort.ily]
other_music_specifier = consort.MusicSpecifier(
    pitch_handler=consort.AbsolutePitchHandler(pitch_specifier='g fs e f'),
    rhythm_maker=rhythmmakertools.EvenDivisionRhythmMaker(
        denominators=[8],
        extra_counts_per_division=(1,),
        ),
    )
other_music_setting = consort.MusicSetting(
    timespan_maker=consort.TaleaTimespanMaker(
        initial_silence_talea=rhythmmakertools.Talea([1], 2),
        silence_talea=rhythmmakertools.Talea([1], 2),
        ),
    v1=other_music_specifier,
    v2=other_music_specifier,
    )
segment_maker = new(
    segment_maker,
    settings=[music_setting, other_music_setting],
    )
show(segment_maker, verbose=False)
</abjad>
\end{comment}

\begin{abjadbookoutput}
\begin{singlespacing}
\vspace{-0.5\baselineskip}
\begin{lstlisting}
>>> other_music_specifier = consort.MusicSpecifier(
...     pitch_handler=consort.AbsolutePitchHandler(pitch_specifier='g fs e f'),
...     rhythm_maker=rhythmmakertools.EvenDivisionRhythmMaker(
...         denominators=[8],
...         extra_counts_per_division=(1,),
...         ),
...     )
>>> other_music_setting = consort.MusicSetting(
...     timespan_maker=consort.TaleaTimespanMaker(
...         initial_silence_talea=rhythmmakertools.Talea([1], 2),
...         silence_talea=rhythmmakertools.Talea([1], 2),
...         ),
...     v1=other_music_specifier,
...     v2=other_music_specifier,
...     )
>>> segment_maker = new(
...     segment_maker,
...     settings=[music_setting, other_music_setting],
...     )
>>> show(segment_maker, verbose=False)
\end{lstlisting}
\noindent\includegraphics[max width=\textwidth,]{assets/lilypond-e6f73fdc68fe0c974a8ca8492d683bdb.pdf}
\end{singlespacing}
\end{abjadbookoutput}

\begin{comment}
<abjad>[stylesheet=../consort.ily]
music_specifier = new(
    music_specifier,
    pitch_handler__pitch_specifier=consort.PitchSpecifier(
        pitch_segments=(
            "c' e' g'",
            "fs' g' a'",
            "b d'",
            ),
        ratio=(1, 2, 3),
        ),
    )
music_setting = consort.MusicSetting(
    timespan_maker=timespan_maker,
    v1=music_specifier,
    v2=music_specifier,
    )
segment_maker = new(segment_maker, settings=[music_setting])
show(segment_maker, verbose=False)
</abjad>
\end{comment}

\begin{abjadbookoutput}
\begin{singlespacing}
\vspace{-0.5\baselineskip}
\begin{lstlisting}
>>> music_specifier = new(
...     music_specifier,
...     pitch_handler__pitch_specifier=consort.PitchSpecifier(
...         pitch_segments=(
...             "c' e' g'",
...             "fs' g' a'",
...             "b d'",
...             ),
...         ratio=(1, 2, 3),
...         ),
...     )
>>> music_setting = consort.MusicSetting(
...     timespan_maker=timespan_maker,
...     v1=music_specifier,
...     v2=music_specifier,
...     )
>>> segment_maker = new(segment_maker, settings=[music_setting])
>>> show(segment_maker, verbose=False)
\end{lstlisting}
\noindent\includegraphics[max width=\textwidth,]{assets/lilypond-f863612768a505f3f1a27901d8908c01.pdf}
\end{singlespacing}
\end{abjadbookoutput}

\begin{comment}
<abjad>[stylesheet=../consort.ily]
music_specifier = new(
    music_specifier,
    pitch_handler__pitch_operation_specifier=consort.PitchOperationSpecifier(
        pitch_operations=(
            pitchtools.PitchOperation((
                pitchtools.Inversion(),
                )),
            None,
            pitchtools.PitchOperation((
                pitchtools.Rotation(-1),
                pitchtools.Transposition(-1),
                ))
            ),
        ratio=(1, 2, 1),
        ),
    )
music_setting = consort.MusicSetting(
    timespan_maker=timespan_maker,
    v1=music_specifier,
    v2=music_specifier,
    )
segment_maker = new(segment_maker, settings=[music_setting])
show(segment_maker, verbose=False)
</abjad>
\end{comment}

\begin{abjadbookoutput}
\begin{singlespacing}
\vspace{-0.5\baselineskip}
\begin{lstlisting}
>>> music_specifier = new(
...     music_specifier,
...     pitch_handler__pitch_operation_specifier=consort.PitchOperationSpecifier(
...         pitch_operations=(
...             pitchtools.PitchOperation((
...                 pitchtools.Inversion(),
...                 )),
...             None,
...             pitchtools.PitchOperation((
...                 pitchtools.Rotation(-1),
...                 pitchtools.Transposition(-1),
...                 ))
...             ),
...         ratio=(1, 2, 1),
...         ),
...     )
>>> music_setting = consort.MusicSetting(
...     timespan_maker=timespan_maker,
...     v1=music_specifier,
...     v2=music_specifier,
...     )
>>> segment_maker = new(segment_maker, settings=[music_setting])
>>> show(segment_maker, verbose=False)
\end{lstlisting}
\noindent\includegraphics[max width=\textwidth,]{assets/lilypond-8353cb721ae26e66270945a62cbb8a67.pdf}
\end{singlespacing}
\end{abjadbookoutput}

\begin{comment}
<abjad>[stylesheet=../consort.ily]
music_specifier = new(
    music_specifier,
    pitch_handler__forbid_repetitions=True,
    )
music_setting = consort.MusicSetting(
    timespan_maker=timespan_maker,
    v1=music_specifier,
    v2=music_specifier,
    )
segment_maker = new(segment_maker, settings=[music_setting])
show(segment_maker, verbose=False)
</abjad>
\end{comment}

\begin{abjadbookoutput}
\begin{singlespacing}
\vspace{-0.5\baselineskip}
\begin{lstlisting}
>>> music_specifier = new(
...     music_specifier,
...     pitch_handler__forbid_repetitions=True,
...     )
>>> music_setting = consort.MusicSetting(
...     timespan_maker=timespan_maker,
...     v1=music_specifier,
...     v2=music_specifier,
...     )
>>> segment_maker = new(segment_maker, settings=[music_setting])
>>> show(segment_maker, verbose=False)
\end{lstlisting}
\noindent\includegraphics[max width=\textwidth,]{assets/lilypond-e8337085639c633444a1e65214aa5493.pdf}
\end{singlespacing}
\end{abjadbookoutput}

\begin{comment}
<abjad>[stylesheet=../consort.ily]
music_specifier = new(
    music_specifier,
    pitch_handler__logical_tie_expressions=(
        consort.ChordExpression(chord_expr=(-2, 0, 2)),
        consort.ChordExpression(chord_expr=(-7, 0, 7)),
        None,
        ),
    )
music_setting = consort.MusicSetting(
    timespan_maker=timespan_maker,
    v1=music_specifier,
    v2=music_specifier,
    )
segment_maker = new(segment_maker, settings=[music_setting])
show(segment_maker, verbose=False)
</abjad>
\end{comment}

\begin{abjadbookoutput}
\begin{singlespacing}
\vspace{-0.5\baselineskip}
\begin{lstlisting}
>>> music_specifier = new(
...     music_specifier,
...     pitch_handler__logical_tie_expressions=(
...         consort.ChordExpression(chord_expr=(-2, 0, 2)),
...         consort.ChordExpression(chord_expr=(-7, 0, 7)),
...         None,
...         ),
...     )
>>> music_setting = consort.MusicSetting(
...     timespan_maker=timespan_maker,
...     v1=music_specifier,
...     v2=music_specifier,
...     )
>>> segment_maker = new(segment_maker, settings=[music_setting])
>>> show(segment_maker, verbose=False)
\end{lstlisting}
\noindent\includegraphics[max width=\textwidth,]{assets/lilypond-1faec952a7b45f54cee2ed62c44d50bc.pdf}
\end{singlespacing}
\end{abjadbookoutput}

\begin{comment}
<abjad>[stylesheet=../consort.ily]
segment_maker = new(
    segment_maker,
    settings=[music_setting, other_music_setting],
    )
show(segment_maker, verbose=False)
</abjad>
\end{comment}

\begin{abjadbookoutput}
\begin{singlespacing}
\vspace{-0.5\baselineskip}
\begin{lstlisting}
>>> segment_maker = new(
...     segment_maker,
...     settings=[music_setting, other_music_setting],
...     )
>>> show(segment_maker, verbose=False)
\end{lstlisting}
\noindent\includegraphics[max width=\textwidth,]{assets/lilypond-9b48c7c7d8ec2ca4e86b9da611fdef17.pdf}
\end{singlespacing}
\end{abjadbookoutput}

\begin{comment}
<abjad>[stylesheet=../consort.ily]
music_specifier = consort.MusicSpecifier(
    pitch_handler=consort.AbsolutePitchHandler(
        pitch_application_rate='division',
        pitch_specifier="c' d' e' f' g' a' b' c''",
        ),
    rhythm_maker=rhythmmakertools.EvenDivisionRhythmMaker(denominators=[16]),
    )
music_setting = consort.MusicSetting(
    timespan_maker=timespan_maker,
    v1=music_specifier,
    v2=music_specifier,
    )
segment_maker = new(segment_maker, settings=[music_setting])
lilypond_file = segment_maker(verbose=False)
consort.annotate(lilypond_file.score)
show(lilypond_file)
</abjad>
\end{comment}

\begin{abjadbookoutput}
\begin{singlespacing}
\vspace{-0.5\baselineskip}
\begin{lstlisting}
>>> music_specifier = consort.MusicSpecifier(
...     pitch_handler=consort.AbsolutePitchHandler(
...         pitch_application_rate='division',
...         pitch_specifier="c' d' e' f' g' a' b' c''",
...         ),
...     rhythm_maker=rhythmmakertools.EvenDivisionRhythmMaker(denominators=[16]),
...     )
>>> music_setting = consort.MusicSetting(
...     timespan_maker=timespan_maker,
...     v1=music_specifier,
...     v2=music_specifier,
...     )
>>> segment_maker = new(segment_maker, settings=[music_setting])
>>> lilypond_file = segment_maker(verbose=False)
>>> consort.annotate(lilypond_file.score)
>>> show(lilypond_file)
\end{lstlisting}
\noindent\includegraphics[max width=\textwidth,]{assets/lilypond-76acdc0bda9feecdd70997204aa7c9c8.pdf}
\end{singlespacing}
\end{abjadbookoutput}

\begin{comment}
<abjad>[stylesheet=../consort.ily]
music_specifier = new(
    music_specifier,
    pitch_handler__pitch_application_rate='phrase',
    )
music_setting = consort.MusicSetting(
    timespan_maker=timespan_maker,
    v1=music_specifier,
    v2=music_specifier,
    )
segment_maker = new(segment_maker, settings=[music_setting])
lilypond_file = segment_maker(verbose=False)
consort.annotate(lilypond_file.score)
show(lilypond_file)
</abjad>
\end{comment}

\begin{abjadbookoutput}
\begin{singlespacing}
\vspace{-0.5\baselineskip}
\begin{lstlisting}
>>> music_specifier = new(
...     music_specifier,
...     pitch_handler__pitch_application_rate='phrase',
...     )
>>> music_setting = consort.MusicSetting(
...     timespan_maker=timespan_maker,
...     v1=music_specifier,
...     v2=music_specifier,
...     )
>>> segment_maker = new(segment_maker, settings=[music_setting])
>>> lilypond_file = segment_maker(verbose=False)
>>> consort.annotate(lilypond_file.score)
>>> show(lilypond_file)
\end{lstlisting}
\noindent\includegraphics[max width=\textwidth,]{assets/lilypond-89244abdb768aa23fa5a7425fab9e539.pdf}
\end{singlespacing}
\end{abjadbookoutput}

\begin{comment}
<abjad>[stylesheet=../consort.ily]
music_specifier = new(
    music_specifier,
    pitch_handler__deviations=(0, 0, '-m2', '+m2'),
    )
music_setting = consort.MusicSetting(
    timespan_maker=timespan_maker,
    v1=music_specifier,
    v2=music_specifier,
    )
segment_maker = new(segment_maker, settings=[music_setting])
lilypond_file = segment_maker(verbose=False)
consort.annotate(lilypond_file.score)
show(lilypond_file)
</abjad>
\end{comment}

\begin{abjadbookoutput}
\begin{singlespacing}
\vspace{-0.5\baselineskip}
\begin{lstlisting}
>>> music_specifier = new(
...     music_specifier,
...     pitch_handler__deviations=(0, 0, '-m2', '+m2'),
...     )
>>> music_setting = consort.MusicSetting(
...     timespan_maker=timespan_maker,
...     v1=music_specifier,
...     v2=music_specifier,
...     )
>>> segment_maker = new(segment_maker, settings=[music_setting])
>>> lilypond_file = segment_maker(verbose=False)
>>> consort.annotate(lilypond_file.score)
>>> show(lilypond_file)
\end{lstlisting}
\noindent\includegraphics[max width=\textwidth,]{assets/lilypond-ff0b1c46ddbd29c5a690fe7c182da0aa.pdf}
\end{singlespacing}
\end{abjadbookoutput}

\subsection{Attachment-handlers}
\label{ssec:attachment-handlers}

Attachment-handlers manage the process of attaching indicators and spanners to
selections of components within a score. They may also evaluate *component
expressions* against selections of components, much like the logical tie
expressions outlined in \autoref{ssec:pitch-handlers}.

Attachment-handlers aggregate *attachment expressions* by associating those
expressions with underscore-delimited string keys. This mechanism, nearly
identical to that employed by music settings, allows attachment expressions --
which may have an arbitrary number of such associations -- to be reconfigured
through templating to add new attachment expressions or overwrite or nullify
specific existing expressions.\footnote{The principle employed by both music
settings and attachment handlers is crucial: named references beat positional
references.}

Attachment expressions pair a *component selector* -- as described in
\autoref{ssec:selectors} -- and an iterable of attachments -- indicators and
spanners -- or component expressions.

Attachment-handlers provide various niceties for instantiation.

\begin{markdown}
-   Attaches things to the score.
-   Aggregates AttachmentExpression instances together.
    -   A bundle of a selector and an iterable of attachments or
        expressions.
    -   Reprise discussion of selectors.
-   Processes the score by voice, then by *phrase*.
    -   Unlike the other handlers, attachment handlers require the entire
        phrase to operate on, and selectors should be designed with that in
        mind.
\end{markdown}

\begin{comment}
<abjad>[stylesheet=../consort.ily]
music_specifier = consort.MusicSpecifier(
    attachment_handler=consort.AttachmentHandler(),
    rhythm_maker=rhythmmakertools.TaleaRhythmMaker(
        extra_counts_per_division=(0, 1),
        talea=rhythmmakertools.Talea([1, 2, 3, 1, 4], 16),
        ),
    )
timespan_maker = consort.TaleaTimespanMaker(
    initial_silence_talea=rhythmmakertools.Talea([0, 1], 4),
    playing_groupings=(1, 2, 2),
    playing_talea=rhythmmakertools.Talea([2, 3], 8),
    silence_talea=rhythmmakertools.Talea([1, 2, 3, 4], 8),
    )
music_setting = consort.MusicSetting(
    timespan_maker=timespan_maker,
    v1=music_specifier,
    v2=music_specifier,
    )
segment_maker = consort.SegmentMaker(
    desired_duration_in_seconds=8,
    discard_final_silence=True,
    permitted_time_signatures=[(2, 4), (5, 16), (3, 4)],
    score_template=templatetools.GroupedRhythmicStavesScoreTemplate(
        staff_count=2,
        with_clefs=True,
        ),
    settings=[music_setting],
    tempo=indicatortools.Tempo((1, 4), 72),
    )
show(segment_maker, verbose=False)
</abjad>
\end{comment}

\begin{abjadbookoutput}
\begin{singlespacing}
\vspace{-0.5\baselineskip}
\begin{lstlisting}
>>> music_specifier = consort.MusicSpecifier(
...     attachment_handler=consort.AttachmentHandler(),
...     rhythm_maker=rhythmmakertools.TaleaRhythmMaker(
...         extra_counts_per_division=(0, 1),
...         talea=rhythmmakertools.Talea([1, 2, 3, 1, 4], 16),
...         ),
...     )
>>> timespan_maker = consort.TaleaTimespanMaker(
...     initial_silence_talea=rhythmmakertools.Talea([0, 1], 4),
...     playing_groupings=(1, 2, 2),
...     playing_talea=rhythmmakertools.Talea([2, 3], 8),
...     silence_talea=rhythmmakertools.Talea([1, 2, 3, 4], 8),
...     )
>>> music_setting = consort.MusicSetting(
...     timespan_maker=timespan_maker,
...     v1=music_specifier,
...     v2=music_specifier,
...     )
>>> segment_maker = consort.SegmentMaker(
...     desired_duration_in_seconds=8,
...     discard_final_silence=True,
...     permitted_time_signatures=[(2, 4), (5, 16), (3, 4)],
...     score_template=templatetools.GroupedRhythmicStavesScoreTemplate(
...         staff_count=2,
...         with_clefs=True,
...         ),
...     settings=[music_setting],
...     tempo=indicatortools.Tempo((1, 4), 72),
...     )
>>> show(segment_maker, verbose=False)
\end{lstlisting}
\noindent\includegraphics[max width=\textwidth,]{assets/lilypond-31ad7eaab600ca94fc940e6c050a8651.pdf}
\end{singlespacing}
\end{abjadbookoutput}

\begin{comment}
<abjad>[stylesheet=../consort.ily]
music_specifier = new(
    music_specifier,
    attachment_handler__accents=consort.AttachmentExpression(
        attachments=Articulation('accent'),
        selector=selectortools.Selector().by_leaves()[0],
        ),
    )
music_setting = new(
    music_setting,
    v1=music_specifier,
    v2=music_specifier,
    )
segment_maker = new(segment_maker, settings=[music_setting])
show(segment_maker,verbose=False)
</abjad>
\end{comment}

\begin{abjadbookoutput}
\begin{singlespacing}
\vspace{-0.5\baselineskip}
\begin{lstlisting}
>>> music_specifier = new(
...     music_specifier,
...     attachment_handler__accents=consort.AttachmentExpression(
...         attachments=Articulation('accent'),
...         selector=selectortools.Selector().by_leaves()[0],
...         ),
...     )
>>> music_setting = new(
...     music_setting,
...     v1=music_specifier,
...     v2=music_specifier,
...     )
>>> segment_maker = new(segment_maker, settings=[music_setting])
>>> show(segment_maker,verbose=False)
\end{lstlisting}
\noindent\includegraphics[max width=\textwidth,]{assets/lilypond-5845a137b0839d77edd44e425c040cdb.pdf}
\end{singlespacing}
\end{abjadbookoutput}

\begin{comment}
<abjad>[stylesheet=../consort.ily]
music_specifier = new(
    music_specifier,
    attachment_handler__tenuti=consort.AttachmentExpression(
        attachments=Articulation('tenuto'),
        selector=selectortools.Selector()
            .by_leaves()[1:]
            .by_logical_tie(pitched=True)[0],
        ),
    )
music_setting = new(
    music_setting,
    v1=music_specifier,
    v2=music_specifier,
    )
segment_maker = new(segment_maker, settings=[music_setting])
show(segment_maker, verbose=False)
</abjad>
\end{comment}

\begin{abjadbookoutput}
\begin{singlespacing}
\vspace{-0.5\baselineskip}
\begin{lstlisting}
>>> music_specifier = new(
...     music_specifier,
...     attachment_handler__tenuti=consort.AttachmentExpression(
...         attachments=Articulation('tenuto'),
...         selector=selectortools.Selector()
...             .by_leaves()[1:]
...             .by_logical_tie(pitched=True)[0],
...         ),
...     )
>>> music_setting = new(
...     music_setting,
...     v1=music_specifier,
...     v2=music_specifier,
...     )
>>> segment_maker = new(segment_maker, settings=[music_setting])
>>> show(segment_maker, verbose=False)
\end{lstlisting}
\noindent\includegraphics[max width=\textwidth,]{assets/lilypond-254b71b1dd161525523b2dfc96ee5b18.pdf}
\end{singlespacing}
\end{abjadbookoutput}

\begin{comment}
<abjad>[stylesheet=../consort.ily]
music_specifier = new(
    music_specifier,
    attachment_handler__slurs=Slur()
    )
music_setting = new(
    music_setting,
    v1=music_specifier,
    v2=music_specifier,
    )
segment_maker = new(segment_maker, settings=[music_setting])
show(segment_maker, verbose=False)
</abjad>
\end{comment}

\begin{abjadbookoutput}
\begin{singlespacing}
\vspace{-0.5\baselineskip}
\begin{lstlisting}
>>> music_specifier = new(
...     music_specifier,
...     attachment_handler__slurs=Slur()
...     )
>>> music_setting = new(
...     music_setting,
...     v1=music_specifier,
...     v2=music_specifier,
...     )
>>> segment_maker = new(segment_maker, settings=[music_setting])
>>> show(segment_maker, verbose=False)
\end{lstlisting}
\noindent\includegraphics[max width=\textwidth,]{assets/lilypond-f0e62c688147028c121f9b53bb698a65.pdf}
\end{singlespacing}
\end{abjadbookoutput}

\begin{comment}
<abjad>[stylesheet=../consort.ily]
music_specifier = new(
    music_specifier,
    attachment_handler__dynamics=consort.DynamicExpression(['f', 'p'])
    )
music_setting = new(
    music_setting,
    v1=music_specifier,
    v2=music_specifier,
    )
segment_maker = new(segment_maker, settings=[music_setting])
show(segment_maker, verbose=False)
</abjad>
\end{comment}

\begin{abjadbookoutput}
\begin{singlespacing}
\vspace{-0.5\baselineskip}
\begin{lstlisting}
>>> music_specifier = new(
...     music_specifier,
...     attachment_handler__dynamics=consort.DynamicExpression(['f', 'p'])
...     )
>>> music_setting = new(
...     music_setting,
...     v1=music_specifier,
...     v2=music_specifier,
...     )
>>> segment_maker = new(segment_maker, settings=[music_setting])
>>> show(segment_maker, verbose=False)
\end{lstlisting}
\noindent\includegraphics[max width=\textwidth,]{assets/lilypond-5134d656fed6cee151678103491f4f07.pdf}
\end{singlespacing}
\end{abjadbookoutput}

\begin{comment}
<abjad>[stylesheet=../consort.ily]
lilypond_file = segment_maker(verbose=False)
consort.annotate(lilypond_file.score)
show(lilypond_file)
</abjad>
\end{comment}

\begin{abjadbookoutput}
\begin{singlespacing}
\vspace{-0.5\baselineskip}
\begin{lstlisting}
>>> lilypond_file = segment_maker(verbose=False)
>>> consort.annotate(lilypond_file.score)
>>> show(lilypond_file)
\end{lstlisting}
\noindent\includegraphics[max width=\textwidth,]{assets/lilypond-6e9d587e5ee37ad6583f5197bed09814.pdf}
\end{singlespacing}
\end{abjadbookoutput}

\subsection{Expressive attachments}
\label{ssec:expressive-attachments}

\begin{markdown}
-   Idiomatic indicators
-   DynamicExpression
-   BowContactSpanner
-   StringContactSpanner
-   Indicators attached as annotations
-   Spanners as a kind of clever post-processing
\end{markdown}

\subsection{Post-processing}
\label{ssec:post-processing}

Score configuration.

Adding a time signature context, as described in
\autoref{ssec:score-post-processing}.

Consort also allows for inserting a LilyPond \texttt{\\include} statement at
the top of any LilyPond file constructed through interpretation which points to
a stylesheet located relative to a path defined on the segment-maker.

By subclasses Consort's segment-maker, individual score packages can cause
their own segment-makers to automatically locate the appropriate stylesheet
file relative to the package where that segment-maker subclass was defined.

A time-signature context is created and inserted at the top of the score if the
score does not already contain such a context. The segment-maker populates this
time-signature context with measures filled with typographic spacer-skips, to
create the appearance of floating time-signature indications.

\begin{markdown}
-   Score configuration
    -   Time signature context
        -   Adding time signatures
        -   Rehearsal marks
        -   Repeat signs
-   LilyPond configuration
    -   Style-sheets
\end{markdown}

\subsection{Voice copying}
\label{ssec:voice-copying}

For reasons related solely to how LilyPond handles various typographic
constructs during typesetting, it may prove necessary to copy certain voices in
the score, maintaining all rhythmic information therein, but changing their
context name and filtering out various spanners and indicators so that only a
specific subset of typographic commands remain.

My score \emph{Invisible Cities (ii): Armilla} employs this voice-copying
technique to create the desired typography.

This is achieved by subclassing Consort's segment-maker.

Voice-copying treats the output of interpretation as \enquote{semantic}, but
not necessarily fully-ready for typesetting.

\section{Persistence \& visualization}

Once interpreted, a segment-maker's illustration may be persisted to disk as
LilyPond syntax for inclusion in other LilyPond files, rendered as a PDF for
viewing, or even serialized for other purposes. Composers study the results of
interpretation, make changes to each segment's specification, and re-interpret
as necessary, a large-scale re-enactment of interactive programming's pervasive
\emph{read-eval-print} loop paradigm.